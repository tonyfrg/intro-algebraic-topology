Nous allons introduire le groupe fondamental, un outil de topologie algébrique, invariant par homéomorphisme. Intuitivement, nous pouvons le voir comme un outil permettant d'étudier le comportement des lacets (voir définition \ref{def:loops}) sur un espace pointé. Cet outil nous permettra de démontrer que la sphère et le tore ne sont pas homéomorphes : l'idée sous-jacente étant que les lacets de la sphère peuvent se contracter en un point (voir figure \ref{fig:lacet-sphere}), tandis que ceux du tore faisant le tour du "trou" en sont incapables (voir figure \ref{fig:lacet-tore}).

\begin{figure}[H]
\centering
\begin{subfigure}[c]{0.6\linewidth}
\centering
\includegraphics[width=0.7\linewidth]{pictures/sphere-intro.png}
\caption{Lacet de la sphère qui se rétracte}
\label{fig:lacet-sphere}
\end{subfigure}
\begin{subfigure}[c]{0.3\linewidth}
\centering
\includegraphics[width=0.7\linewidth]{pictures/tore-intro.png}
\caption{Lacet "coincé" du tore}
\label{fig:lacet-tore}
\end{subfigure}
\end{figure}

Ce chapitre est décomposé en trois grandes sections : \begin{enumerate}
    \item La construction du groupe fondamental, où nous démontrons notamment qu'il possède une structure de groupe (d'où son nom). Nous démontrons ensuite qu'il s'agit d'un invariant par homéomorphisme, et par conséquent qu'il permet de discriminer les espaces. Nous démontrons ainsi que la sphère et le tore ne peuvent être homéomorphes.
    \item Le groupe fondamental du cercle. Sa caractérisation n'est pas si trivial qu'il n'y paraît, et implique notamment l'introduction des revêtements. Suite à ce résultat, nous arrivons à caractériser plusieurs autres groupes fondamentaux.
    \item Les revêtements possèdent un lien encore plus puissant avec les groupes fondamentaux. En particulier, il existe une correspondance de Galois entre les sous-groupes du groupe fondamental et les revêtements de l'espace à isomorphisme près. La démonstration de cette correspondance n'étant pas triviale, elle est découpée en plusieurs parties. 
\end{enumerate}

Sur l'ensemble du chapitre, nous avons semé des exemples simples, apportant une illustration aux propos énoncés.

\section{Un groupe fondamental ?}

Dans cette première section, nous allons définir le groupe fondamental, et démontrer qu'il possède une structure de groupe. Pour cela, nous nous introduisons les lacets, et s'intéresser à eux selon une relation d'équivalence, nommée homotopie.

\begin{definition}\label{def:loops}
Un \emph{lacet}, ou une \emph{boucle}, est un chemin ayant le même point de départ et d'arrivée. Plus formellement, il s'agit d'une application~$\gamma:[0,1]\to X$ telle que~$\gamma(0)=\gamma(1)$. Nous appelons \emph{point de base} le point de départ et d'arrivée du lacet.
\end{definition}

Un lacet est un chemin, il possède donc également une direction, allant de $\gamma(0)$ à $\gamma(1)$.

\begin{exemple}
En considérant le cercle $\s{1}$ dans le plan complexe, le chemin défini par $t\mapsto e^{2i\pi t}$ est un lacet ayant pour point de base~$(1,0)$.
\end{exemple}

L'exemple suivant est important, car introduit le concept de \emph{polygone fondamental}, omniprésent en topologie algébrique, et par conséquent dans ce mémoire. Cette représentation est dite intrinsèque, car ne dépend d'aucun espace environnant. Celle-ci est aux antipodes de leurs plongements dans certains espaces euclidiens~$\bb{R}^n$, même s'il existe un homéomorphisme entre les deux représentations \cite{Homeo-article}.

\begin{exemple}
La figure \ref{tkz:mobius} est la représentation du polygône fondamental du ruban de Mobiüs, un carré avec une identification de deux côtés opposés via une torsion (les deux flèches vont dans deux sens différents).
\begin{figure}[H]
\centering
\begin{subfigure}[b]{0.45\linewidth}
\centering
\begin{tikzpicture}
    % Background color
    \fill[gray!30] (0,0) rectangle (4,4);

    % Top arrow with label A
    \draw[thick, red, ->, >=stealth'] (0,4) -- (4,4) node[midway, below] {\textbf{A}};

    % Bottom arrow with label A
    \draw[thick, red, ->, >=stealth'] (4,0) -- (0,0) node[midway, above] {\textbf{A}};

    % Outer rectangle
    \draw[black, line width=.0mm] (0,0) rectangle (4,4);
\end{tikzpicture}
\caption{Ruban de Mobius}
\label{tkz:mobius}
\end{subfigure}
\begin{subfigure}[b]{0.45\linewidth}
\centering
\begin{tikzpicture}
    % fond gris
    \fill[gray!30] (0,0) rectangle (4,4);
    %flèche du haut
    \draw[thick, red, ->, >=stealth'] (0,4) -- (4,4) node[midway, below] {\textbf{A}};
    %flèche du bas
    \draw[thick, red, ->, >=stealth'] (4,0) -- (0,0) node[midway, above] {\textbf{A}};
    %carré
    \draw[black, line width=.0mm] (0,0) rectangle (4,4);
    
    %gamma'
    \draw[thick, black, ->] (3,0) -- (2.5,1);
    \draw[thick, black] (2.5,1) -- (2,2) node[near start, left] {$\gamma'$};
    \draw[thick, black, ->] (2,2) -- (1.5,3);
    \draw[thick, black] (1.5,3) -- (1,4);

    %gamma
    \draw[thick, ->] (3.4, 2) arc[start angle=0, end angle=360, radius=0.7] node[right] {$\gamma$};
    % Point at (2,2)
    \filldraw[black] (2,2) circle (2pt) node[right] {$x_0$};
\end{tikzpicture}
\caption{Ruban de Mobius et deux lacets}
\label{tkz:mobius-loops}
\end{subfigure}
\end{figure}

Nous pouvons tracer une infinité de lacet sur le ruban, mais deux grandes "familles" en ressortent, illustrés dans la figure \ref{tkz:mobius-loops} : les lacets qui restent dans l'intérieur du carré (comme le lacet $\gamma$), et les lacets qui utilisent l'identification $A$ des côtés (comme le lacet $\gamma'$).
\end{exemple}

\subsection{Homotopie}\label{sec:homotopy}

L'homotopie est la formalisation de l'idée de la déformation d'objets, en particuliers des chemins et lacets sur un espace. Le concept de lacet qui se rétracte (présenté en introduction) devient cohérent d'un point de vue topologique.

\begin{definition}\label{def:homotopy-paths}
Soient $x_0, x_1$ deux points de l'espace $X$, et soient $\gamma_0$ et $\gamma_1$ deux chemins partageant les mêmes extrémités $x_0$ et $x_1$. Une \emph{homotopie} entre $\gamma_0$ et $\gamma_1$ est une application $\Gamma:[0,1]^2\to X$, définie par $\Gamma(t,s)=\gamma_t(s)$, continue en $t$ et telle que, pour tout $t\in[0,1]$, $\gamma_t(0)=x_0$ et $\gamma_t(1)=x_1$.

Intuitivement, cela revient à dire qu'il existe une famille $\gamma_t$ de chemin fixant les extrémités, et continue en $t$. Nous disons alors que $\gamma_0$ est homotope à $\gamma_1$, ou bien que $\gamma_0$ et $\gamma_1$ sont homotopes, et l'on note $\gamma_0\simeq\gamma_1$.
\end{definition}

\begin{figure}[H]
    \centering
    \begin{tikzpicture}
    % Points A et B
    \filldraw[black] (0,0) circle (2pt) node[above left] {$x_0$};
    \filldraw[black] (6,0) circle (2pt) node[above right] {$x_1$};
    % t=0
    \draw[thick, ->, >=stealth] (0,0) .. controls (1.5,0.8) .. (3,1) node[near end, above] {$\gamma_0$}; 
    \draw[thick] (3,1) .. controls (4.5,0.8) .. (6,0);
    %t=.25
    \draw[dashed, ->, >=stealth'] (0,0) .. controls (1.5,0.4) .. (3,0.5); 
    \draw[dashed] (3,0.5) .. controls (4.5,0.4) .. (6,0);
    %t=0.5
    \draw[thick, dashed, ->, >=stealth'] (0,0) -- (3,0)  node[near end, above] {$\gamma_{\frac{1}{2}}$}; 
    \draw[thick, dashed] (3,0) -- (6,0);
    %t=.75
    \draw[dashed, ->, >=stealth'] (0,0) .. controls (1.5,-0.4) .. (3,-0.5); 
    \draw[dashed] (3,-0.5) .. controls (4.5,-0.4) .. (6,0);
    % t=1
    \draw[thick, ->, >=stealth'] (0,0) .. controls (1.5,-0.8) .. (3,-1) node[near end, below] {$\gamma_1$}; 
    \draw[thick] (3,-1) .. controls (4.5,-0.8) .. (6,0);
\end{tikzpicture}
    \caption{Homotopie entre les chemins $\gamma_0$ et $\gamma_1$}
    \label{tkz:path-homotopy}
\end{figure}

\begin{exemple}
Dans l'espace euclidien $\bb{R}^n$, il existe toujours une homotopie linéaire entre deux chemins avec les mêmes extrémités. En effet, en notant $\gamma_0$ et $\gamma_1$ deux chemins d'extrémités communes $x_0$ et $x_1$, nous avons l'homotopie suivante, pour~$t\in[0,1]$ :  $\gamma_t(s)=(1-t)\gamma_0(s)+t\gamma_1(s)$. Il est facile de vérifier que $\gamma_t(0)=x_0$ et~$\gamma_t(1)=x_1$, pour tout $t\in[0,1]$.
\end{exemple}

Plutôt que de s'intéresser à l'ensemble des lacets, nous allons préférer les étudier à homotopie près. D'une part, il est possible dans certains cas de passer d'une infinité de lacets à un nombre fini de classes d'équivalences. Et d'autre part, les classes d'homotopie de lacets possèdent une structure naturelle de groupe. C'est ce que nous allons montrer par la suite.

\begin{proposition}
La relation d'homotopie entre chemins est une relation d'équivalence. Nous noterons $[\gamma]$ la classe d'équivalence de $\gamma$ sous la relation d'homotopie, et nous appellée \emph{classe d'homotopie} de $\gamma$.
\end{proposition}

En rappellant qu'une relation d'équivalence est définie comme étant réflexive, symétrique et associative, la démonstration consiste simplement à montrer successivement que les trois propriétés sont vérifiées par la relation d'homotopie.
\begin{proof}
\textit{(Réfléxivité)} Pour tout chemin $\gamma$ sur $X$, il existe l'homotopie constante $\gamma_t(s)=\gamma(s)$. Nous en conluons que $\gamma\simeq\gamma$.

\textit{(Symétrie)} Pour toute homotopie $\gamma_t(s)$ entre $\gamma$ et$\gamma'$, il est clair que l'homotopie $\gamma_{1-t}$ permet d'aller de~$\gamma'$ vers $\gamma$. Nous en concluons que $\gamma'\simeq\gamma$.

\textit{(Associativité)} Soient $\gamma\simeq\gamma'$ et $\gamma'\simeq\gamma''$. En notant $\gamma_t$ l'homotopie entre $\gamma$ et $\gamma'$ et $\gamma_t'$ l'homotopie entre $\gamma'$ et~$\gamma''$, nous pouvons  construire l'homotopie $\gamma''_t$ entre $\gamma$ et $\gamma''$, correspondant aux deux homotopies l'une après l'autre : \[\gamma''_t(s)= \left\{\begin{matrix}
\gamma_{2t}(s),&\text{si }t\in[0,\frac{1}{2}], \\
\gamma'_{2t-1}(s),&\text{si }t\in[\frac{1}{2},1].
\end{matrix}\right.\]Celle-ci est continue par définition des deux autres homotopies, en particulier en $1\over 2$ du fait qu'elle se recoupe en le même chemin : $\gamma_1=\gamma'=\gamma'_0$. Nous en concluons que $\gamma\simeq\gamma''$.
\end{proof}

Les lacets étant des chemins, nous pouvons considérer les classes d'homotopie de ceux-ci.

\subsection{Loi de composition interne}

Désormais, pour pouvoir parler de groupe, nous avons besoin d'un opérateur entre les classes d'homotopie. Nous allons alors partir d'une loi de composition interne (LCI) sur les chemins qui soit compatible avec la relation d'équivalence d'homotopie.

\begin{wrapfigure}{r}{0.3\textwidth}
    \centering
    \begin{tikzpicture}
    \filldraw[black] (0,0) circle (2pt) node[above left] {$x_0$};
    \filldraw[black] (2,0.5) circle (2pt) node[above] {$x_1$};
    \filldraw[black] (3,1.7) circle (2pt) node[above left] {$x_2$};
    \draw[thick, ->, >=stealth, red] (0,0) .. controls (0.5,0) .. (1,0.05) node[near start, above] {$\gamma$}; 
    \draw[thick, red] (1,0.05) .. controls (1.5,0.2) .. (2,0.5) ; 
    \draw[thick, blue, ->, >=stealth'] (2,0.5) .. controls (2.5,0.6) .. (3,1) ; 
    \draw[thick, blue] (3,1) .. controls (3.1,1.2) .. (3,1.7) node[near end, right] {$\gamma'$};
    \node[violet] at (2,0.2) {$\gamma\cdot\gamma'$};
    \end{tikzpicture}
\end{wrapfigure}

\phantom{}
\begin{definition}
Pour $\gamma,\gamma'$ deux chemins sur un espace $X$ tels que~$\gamma(1)=\gamma'(0)$, nous pouvons définir le lacet $\gamma\cdot\gamma'$ comme étant le chemin résultat de la \emph{concaténation} de $\gamma$ et $\gamma'$. Formellement, nous le définissons comme suit : \[\gamma\cdot \gamma'(s)=\left\{\begin{matrix}
\gamma(2s)&\text{si }0\leq s\leq \frac{1}{2}\\ 
\gamma'(2s-1)&\text{si }\frac{1}{2}\leq s\leq 1.
\end{matrix}\right.\]

En particulier pour les lacets avec le même point de base, la concaténation est une LCI.
\end{definition}

%En effet, la concaténation de deux lacets donne un lacet qui passera sur le point de base à mi-parcours. Nous pouvons visualler cela avec le chiffre 8, qui est concaténation d'un lacet supérieur et un lacet inférieur, où le point de base est le point d'intersection des deux lacets.

\begin{proposition}
La LCI de la concaténation de lacets est compatible avec la relation d'homotopie. Nous définissons ainsi une LCI sur les classes d'homotopie de lacets, telle que pour $\gamma$ et $\gamma'$ deux lacets ayant le même point de base, on ait $[\gamma][\gamma']=[\gamma\cdot\gamma']$.
\end{proposition}

%La démonstration consiste à vérifier si, pour deux paires de lacets homotopes, la concaténation d'éléments de chaque pair est homotope à la concaténation des deux autres. Pour cela, nous allons construire l'homotopie entre les deux concaténations, à partir des homotopies de base.

Nous utiliserons le terme de concaténations lorsqu'il s'agit de lacets, et de produit lorsqu'il s'agit de classes.


\begin{proof}
Soient $\gamma\simeq\gamma'$ et $\zeta\simeq\zeta'$ quatres lacets ayant un même point de base $x_0\in X$. Il nous suffira de montrer que $\gamma\cdot\zeta\simeq\gamma'\cdot\zeta'$.

Notons respectivement $\gamma_t$ et $\zeta_t$ les homotopies allant de $\gamma$ vers $\gamma'$ et de $\zeta$ vers~$\zeta'$, et considérons l'application obtenue par concaténation de ces homotopies : \begin{equation}\label{eq:homotopy-concatenate}
\tau_t(s)=\gamma_t\cdot\zeta_t(s)=\left\{\begin{matrix}
\gamma_t(2s)&\text{si }0\leq s\leq \frac{1}{2}\\ 
\zeta_t(2s-1)&\text{si }\frac{1}{2}\leq s\leq 1.
\end{matrix}\right.
\end{equation}

Montrons que $\tau_t$ est une homotopie entre $\gamma\cdot\zeta$ et $\gamma'\cdot\zeta'$. D'une part, les points de départs et d'arrivées sont respectivement $\tau_0=\gamma_0\cdot\zeta_0=\gamma\cdot\zeta$, et $\tau_1=\gamma_1\cdot\zeta_1=\gamma'\cdot\zeta'$. Pour ce qui est des extrémités des lacets, ceux-ci restent constant en $x_0$ par définition de $\gamma_t$ et $\zeta_t$ : $\gamma_t\cdot\zeta_t(0)=\gamma_t(0)=x_0$ et $\gamma_t\cdot\zeta_t(1)=\zeta_t(1)=x_0$. Enfin la continuité de $\tau_t$ est évidement due à celle des deux autres homotopies.

Nous venons de prouver qu'il existe une homotopie entre $\gamma\cdot\zeta$ et $\gamma'\cdot\zeta'$, ce qui les rend homotopes.
\end{proof}

\begin{definition}
L'ensemble des classes d'homotopie de lacets de $X$ basés en $x_0\in X$ est appelé le \emph{groupe fondamental} de $X$ en $x_0$, ou sur l'espace pointé $(X,x_0)$, et on note $\pi_1(X,x_0)$.
\end{definition}

Notre objectif est de démontrer que le groupe fondamental muni du produit sur les classes de lacets possède une structure de groupe. Nous commençons par démontrer qu'il existe un élément neutre.

\begin{lemma}
L'élément neutre pour le produit de classes de lacets est la classe du lacet constant $c$ en le point de base.
\end{lemma}

Notre démonstration sera constructive : nous expliciterons les homotopies entre les lacets. A notre connaissance, il n'existe pas de démonstration entrant autant dans les détails, pour la simple raison de la trivialité du raisonnement.

\begin{proof}
Soit~${\alpha\in\pi_1(X,x_0)}$ une classe d'homotopie, avec $\gamma$ un représentant. Nous construirons deux homotopies : l'une entre $\gamma\cdot c$ et $\gamma$, et l'autre entre $c\cdot\gamma$ et $\gamma$. Cela permet de déduire que $\alpha\cdot [c]\simeq\alpha$ et~$[c]\cdot \alpha\simeq \alpha$, et donc démontrer que $[c]$ est le neutre pour le produit.

On construit tout d'abord l'application permettant d'"effacer" homotopiquement le lacet constant à $\gamma\cdot c$ : \[t\in[0,1]\mapsto\gamma_t(s)=\left\{\begin{matrix}
\gamma\big(2s-t\big)&\text{si }0\leq s\leq \frac{1}{2}(1+t)\\ 
c(0)=x_0&\text{sinon}.
\end{matrix}\right.\]

L'application est continue en $s$, en particulier pour $s=\frac{1}{2}(1+t)$, du fait que $\gamma(2s-t)=\gamma(1)=x_0$. L'application est continue en $t$, et les extrémités sont clairement conservées. Il s'agit alors d'une homotopie entre $\gamma_0=\gamma\cdot c$ et $\gamma_1=\gamma$.

De manière similaire, nous pouvons construire l'application permettant d'effacer homotopiquement le lacet constant à $c\cdot\gamma$ : \[t\in[0,1]\mapsto\gamma'_t(s)=\left\{\begin{matrix}
c\left(0\right)=x_0&\text{si }0\leq s\leq \frac{1}{2}(1-t)\\ 
\gamma\left(\frac{2s+t-1}{t+1}\right)&\text{si }\frac{1}{2}(1-t)\leq s\leq 1.
\end{matrix}\right.\]

Montrons qu'il s'agit d'une homotopie, en commençant par la continuité. Pour $s=\frac{1}{2}(1-t)$, on a ${2s+t-1=(1-t)+t-1=0}$, d'où $\gamma\big(\frac{2s+t-1}{t+1}\big)=x_0$. Cela montre que notre application est continue. De même que précédemment, il est facile de vérifier que cela constitue une homotopie entre les lacets~$c\cdot\gamma$ et $\gamma$.

Nous en concluons que la classe d'homotopie $[c]$ est l'élément neutre pour le produit de classes.
\end{proof}

Nous allons désormais démontrer que la structure du groupe fondamental justifie une telle terminologie.

\begin{theorem}
L'ensemble $\pi_1(X,x_0)$ muni du produit sur les classes de lacets possède une structure de groupe, avec le lacet constant comme élément neutre.
\end{theorem}

Nous allons démontrer ce résultat de manière classique, en prouvant l'existance de l'inverse, et en prouvant l'associativité. Encore une fois, cette preuve est constructive, et ne se fait que rarement (voire jamais).

\begin{proof}
On considère le groupe fondamental $\pi_1(X,x_0)$ pour un espace $X$ et un point $x_0\in X$. Nous avons déjà démontré dans le lemme précédent que le lacet constant est l'élément neutre.

\bigskip \textit{(Inverse)} Soit $\alpha\in\pi_1(X,x_0)$ et soit $\gamma$ un lacet représentant de la classe. Notons $\overline{\gamma}(s)=\gamma(1-s)$ le lacet inverse de $\gamma$. Nous allons montrer que $\gamma\cdot\overline{\gamma}\simeq c$ et $\overline{\gamma}\cdot\gamma\simeq c$, ce qui suffit pour démontrer l'existance de l'inverse : $\alpha\cdot [\overline{\gamma}]=[c]$ et $[\overline{\gamma}]\cdot \alpha=[c]$.

Pour cela, on considère une première application, dans le but de vérifier $\gamma\cdot \overline{\gamma}\simeq c$ : \[t\in[0,1]\mapsto\gamma_t(s)=\left\{\begin{matrix}
\gamma\big(2s(1-t)\big)&\text{si }0\leq s\leq \frac{1}{2}\\ 
\overline{\gamma}\big((2s-1)(1-t)+t\big)&\text{si }\frac{1}{2}\leq s\leq 1.
\end{matrix}\right.\]Montrons qu'il s'agit d'une homotopie, en commençant par la continuité. La consitnuité en $t$ est évidente car n'agit seulement de chemin linéaire sur $[0,1]$. La continuité en $s$ est de même simple, mais vérifions la jonction en $1\over2$ : d'un côté, nous avons $\gamma(2s(1-t))=\gamma(1-t)$, et de l'autre $\overline{\gamma}((2s-1)(1-t)+t)=\overline{\gamma}(t)=\gamma(1-t)$. Aux extrémités de $s$, nous avons $\gamma_t(0)=\gamma(0)=x_0$, et $\gamma_t(1)=\overline{\gamma}(1-t+t)=\gamma(0)=x_0$. Il s'agit finalement d'une homotopie entre $\gamma_0=\gamma\cdot \overline{\gamma}$ et $\gamma_1=\gamma(0)\cdot\overline{\gamma}(0)=c$  Ensuite, il est facile de vérifier la continuité en $t$, et que l'on a $\gamma_0=\gamma\cdot\overline{\gamma}$ et $\gamma_1=c$. Nous pouvons alors en déduire que cette application est une homotopie entre les lacets $\gamma\cdot\overline{\gamma}$ et $c$.

\bigskip \textit{(Associativité)} Il nous reste à démontrer l'associativité. Soient $\alpha,\beta,\delta\in\pi_1(X,x_0)$, avec comme représentants respectifs~$\gamma,\zeta,\theta$. Il nous suffira de  montrer que l'on a~$\gamma\cdot(\zeta\cdot\theta)\simeq(\gamma\cdot\zeta)\cdot\theta$. Nous commençons par écrire explicitement la définition de ces lacets : \[\begin{split}
\gamma\cdot(\zeta\cdot\theta)(s)&=\left\{\begin{matrix}
    \gamma(2s)&\text{si }0\leq s\leq\frac{1}{2}\\
    \zeta\cdot\theta(2s-1)&\text{si }\frac{1}{2}\leq s\leq 1
\end{matrix}\right.\\
&=\left\{\begin{matrix}
    \gamma(2s)&\text{si }0\leq s\leq\frac{1}{2}\\
    \zeta(4s-2)&\text{si }\frac{1}{2}\leq s\leq\frac{3}{4}\\
    \theta(4s-3)&\text{si }\frac{3}{4}\leq s\leq1,
\end{matrix}\right.
\end{split}\]et\[\begin{split}
(\gamma\cdot\zeta)\cdot\theta(s)&=\left\{\begin{matrix}
    \gamma\cdot\zeta(2s)&\text{si }0\leq s\leq\frac{1}{2}\\
    \theta(2s-1)&\text{si }\frac{1}{2}\leq s\leq 1
\end{matrix}\right.\\
&=\left\{\begin{matrix}
    \gamma(4s)&\text{si }0\leq s\leq\frac{1}{4}\\
    \zeta(4s-1)&\text{si }\frac{1}{4}\leq s\leq\frac{1}{2}\\
    \theta(2s-1)&\text{si }\frac{1}{2}\leq s\leq1.
\end{matrix}\right.
\end{split}\]On considère l'application suivante, dans l'objectif d'obtenir une homotopie entre les deux lacets explicités plus haut : \[t\in[0,1]\mapsto h_t(s)=\left\{\begin{matrix}
\gamma\left(\frac{4s}{2-t}\right)&\text{si }0\leq s\leq \frac{1}{4}(2-t)\\
\zeta\left(4s-2+t\right)&\text{si }\frac{1}{4}(2-t)\leq s\leq \frac{1}{4}(3-t)\\
\theta\left(\frac{4s-4}{1+t}+1\right)&\text{si }\frac{1}{4}(3-t)\leq s\leq1.
\end{matrix}\right.\]Montrons désormais qu'il s'agit d'une homotopie. La continuité en $t$ est évidente, et celle en $s$ demande une vérification sur les jonctions. En $\frac{1}{4}(2-t)$, nous avons l'égalité $\gamma\left(\frac{4s}{2-t}\right)=\gamma(1)=x_0$ d'une part, et $\zeta(4s-2+t)=\zeta(0)=x_0$ de l'autre. En $s=\frac{1}{4}(3-t)$, nous avons $\zeta(4s-2+t)=\zeta(1)=x_0$ d'une part et $\theta(\frac{4s-4}{1-t}+1)=\theta(0)=x_0$. De plus pour les extrémités, nous avons $h_t(0)=\gamma(\frac{0}{2-t})=x_0$ et $h_t(1)=\theta(\frac{4-4}{1+t}+1)=\theta(1)=x_0$. Nous pouvons en conclure que $h_t$ définie une homotopie entre $h_0$ et $h_1$, mais quels sotn ces lacets ?Par une simple substitution, nous obtenons les résultats suivants : \[h_0(s)=\left\{\begin{matrix}
\gamma\left(\frac{4s}{2-0}\right)=\gamma\left(2s\right) & 0\leq s\leq \frac{1}{2} \\
\zeta(4s-2+0) & \frac{1}{2}\leq s\leq \frac{3}{4} \\
\theta\left(\frac{4s-4}{1+0}\right)=\theta(4s-3) & \frac{3}{4} \leq s\leq 1,
\end{matrix}\right.\]et \[h_1(s)=\left\{\begin{matrix}
\gamma\left(\frac{4s}{2-1}\right)=\gamma(4s) & 0\leq s\leq \frac{1}{4} \\
\zeta(4s-2+1)=\zeta(4s-1) & \frac{1}{4}\leq s\leq \frac{1}{2} \\
\theta\left(\frac{4s-4}{1+1}\right)=\theta(2s-1) & \frac{1}{2} \leq s\leq 1.
\end{matrix}\right.\]Nous en concluons qu'il s'agit d'une homotopie entre $\gamma\cdot(\zeta\cdot \theta)$ et $(\gamma\cdot \zeta)\cdot \theta$, ce qui permet de démontrer l'associaticité sur le produit des classes d'homotopies.

\bigskip Nous achevons ainsi la démonstration en concluant que le groupe fondamental $\pi_1(X,x_0)$ est un groupe.
\end{proof}

Intuitivement, nous venons de démontrer que les classes d'homotopie ne se préoccupe pas du "temps de parcours" du lacet, seulement du trajet (à homotopie près évidement). Si l'on parcourt un lacet dans un sens puis dans l'autre sens, nous considérons en terme d'homotopie que cela revient à ne pas bouger.

\subsection{Propriétés élémentaires}

Désormais que nous savons qu'il existe une structure de groupe sur un espace, il serait bien de l'utiliser. Avant de voir un exemple concret de calcul de groupe, nous allons voir quelques propriétés de base, en commençant par une définition.

\begin{definition}
Un espace est dit \emph{simplement connexe} si son groupe fondamental est trivial.
\end{definition}

Le terme "simplement" réfère directement à l'ensemble des classes d'homotopies, qui possède un seul élément.

\begin{proposition}
Tout sous-espace convexe $X\subset\bb{R}^n$ est simplement connexe.
\end{proposition}

Nous avons déjà vu dans l'exemple à al suite de la définition \ref{def:homotopy-paths} qu'il existe une homotopie (linéaire) entre tout chemin partageant les mêmes extrémités. Ce résultat reste évidement vraie dans le cas des lacets, ce qui rend l'ensemble des classes réduit à un élément : d'où la proposition !

\begin{proposition}
Soit $X$ un espace, et soient $x_0,x_1\in X$ deux points reliés par un chemin. Les groupes fondamentaux $\pi_1(X,x_0)$ et $\pi_1(X,x_1)$ sont isomorphes.
\end{proposition}



\begin{proof}
Soit $\zeta$ le chemin entre $x_0=\zeta(0)$ et $x_1=\zeta(1)$. Pour un lacet~$\gamma$ de $X$ ayant pour point de base $x_0$, nous avons $\zeta\cdot\gamma\cdot\overline{\zeta}$ qui est un lacet basé en~$x_1$. Nous pouvons alors définir l'application~$\Phi:\pi_1(X,x_0)\to\pi_1(X,x_1)$, qui envoie~$[\gamma]$ sur $[\zeta\cdot\gamma\cdot\overline{\zeta}]$. Montrons que cette application est un morphisme de groupes, puis qu'elle admet un morphisme réciproque, de telle sorte que ce soit un isomorphisme.

Nous devons tout d'abord montrer que l'application est bien définie, c'est à dire pour $\gamma\simeq\gamma'$ deux lacets de $(X,x_0)$, nous avons $\zeta\cdot\gamma\cdot\overline{\zeta}\simeq\zeta\cdot\gamma'\cdot\overline{\zeta}$. En notant~$\gamma_t$ l'homotopie entre $\gamma$ et $\gamma'$, nous avons $\zeta\cdot\gamma_t\cdot\overline{\zeta}$ qui forme une homotopie entre $\zeta\cdot\gamma\cdot\overline{\zeta}$ et $\zeta\cdot\gamma'\cdot\overline{\zeta}$.

\bigskip Soient $\alpha,\alpha'\in\pi_1(X,x_0)$, montrons que $\Phi(\alpha\alpha')=\Phi(\alpha)\Phi(\alpha')$. Soient $\gamma,\gamma'$ deux lacets de $(X,x_0)$, représentants respectifs de $\alpha$ et $\alpha'$. Nous avons : \[\begin{split}
\Phi([\gamma][\gamma'])&=\Phi([\gamma\cdot\gamma'])\\
&=[\zeta\cdot\gamma\cdot\gamma'\cdot\overline{\zeta}]\\
&=[\zeta\cdot\gamma\cdot\overline{\zeta}\cdot\zeta\cdot\gamma'\cdot\overline{\zeta}]\\
&=[\zeta\cdot\gamma\cdot\overline{\zeta}][\zeta\cdot\gamma'\cdot\overline{\zeta}]\\
&=\Phi([\gamma])\Phi([\gamma']).
\end{split}\] Nous pouvons alors en déduire qu'il s'agit d'un morphisme. Désormais, on considère l'application allant de $\pi_1(X,x_1)$ vers $\pi_1(X,x_0)$ par~$\Psi:[\gamma]\mapsto[\overline{\zeta}\cdot\gamma\cdot\zeta]$. Pour la même raison que pour $\Phi$, il s'agit d'un morphisme bien définit. Il est évident que $\Phi$ et $\Psi$ sont des morphismes réciproques l'un par rapport à l'autre, puisque~$[\zeta\cdot\overline{\zeta}]=[c]$ et~$[\overline{\zeta}\cdot\zeta]=[c]$. Nous pouvons en conclure que les deux groupes sont isomorphes.
\end{proof}
\begin{remark}
Autrement dit, le groupe fondamental d'un espace connexe par arc ne dépend pas du point choisi. Nous pouvons alors négliger la notation du point de base pour de tel espace.
\end{remark}

\subsubsection{Morphisme induit}

Le premier résultat qui nous intéresse est celui de l'invariance du groupe fondamental par homéomorphisme. Cela vient du fait de la bonne propriété que possède le groupe fondamental sur les inductions de morphismes :

\begin{definition}
Soit $f:(X,x_0)\to(Y,y_0)$ une application entre deux espaces (le point $x_0$ est envoyé sur $y_0$ via $f$). Le \emph{morphisme de groupes induit} par $f$ se défini par $f_\ast:\pi_1(X,x_0)\to\pi_1(Y,y_0)$, qui à un élément $[\gamma]\in\pi_1(X,x_0)$ renvoie~$[f\gamma]\in\pi_1(Y,y_0)$. 
\end{definition}

En théorie des catégories \ref{chap:categories}, on dit que le groupe fondamental est un \emph{foncteur} entre la catégorie des espaces pointés et la catégorie des groupes. Cela signifie qu'il vérifie le théorème suivant.

\begin{theorem}
Nous avons $id_\ast=id$, et pour $f:Y\to Z, g:X\to Y$,  l'égalité $(fg)_\ast=f_\ast g_\ast$. En théorie des catégories, nous disons que le groupe fondamental est un foncteur \emph{covariant}.
\end{theorem}
\begin{proof}
Le morphisme induit par l'identité $id:X\to X$, est défini par $[\gamma]\mapsto[id(\gamma)]=[\gamma]$, ce n'est rien d'autre que l'identité.

En reprenant les notations de l'énoncé, nous avons l'égalité suivante : \[f_\ast(g_\ast([\gamma]))=f_\ast([g\gamma])=[fg\gamma]=(fg)_\ast([\gamma]).\]
\end{proof}



\begin{theorem}
Soit $X$ et $Y$ deux espaces homéomorphes, et soit $x_0\in X$. L'homéomorphisme~$f:X\to Y$ induit un isomorphisme entre les groupes fondamentaux $f_\ast:\pi_1(X,x_0)\to\pi_1(Y,f(x_0))$.
\end{theorem}
\begin{proof}
Du fait que $f$ soit une bijection, elle admet une inverse. Cette inverse induit une inverse pour $f_\ast$, d'où le fait que les groupes fondamentaux soient isomorphes.
\end{proof}

La contraposée est plus intéressante, puisqu'elle permet d'affirmer grâce au groupe fondamental si deux groupes sont non homéomorphes.

\begin{corollary}
Si deux espaces $X$ et $Y$ ne possèdent pas le même groupe fondamental, alors ils ne sont pas homéomorphes.
\end{corollary}
\begin{remark}
Attention toutefois, la réciproque n'est pas vraie, comme nous pouvons le voir juste après.
\end{remark}

\begin{proposition}\label{prop:eq-homo-same-gr-fund}
Une équivalence d'homotopie $\phi:X\to Y$ induit un isomorphisme  entre$\pi_1(X,x_0)$ et~$\pi_1(Y,\phi(x_0))$, pour tout $x_0\in X$.
\end{proposition}
\begin{proof}

\end{proof}

\begin{proposition}
Si un espace $X$ se rétracte en $A$, alors l'inclusion induit un morphisme injectif~$i_\ast:\pi_1(A,x_0)\to\pi_1(X,x_0)$. Si c'est une déformation par rétraction, alors le morphisme induit est un isomorphisme.
\end{proposition}
\begin{proof}
Soient $r:X\to A$ et $i:A\hookrightarrow X$, tels que nous avons $ri=id_A$, ce qui induit~$r_\ast i_\ast=id$. Du fait qu'avoir un inverse à gauche implique être injectif, on en déduit que $i_\ast$ est un morphisme injectif.

Une déformation par rétraction est un cas particulier d'équivalence d'homotopie, nous pouvons alors utiliser la proposition précédente.
\end{proof}

\begin{comment}
\begin{remark}
Le fait que l'on obtiennent des isomorphismes de groupes fondamentaux sur des espaces non homéomorphes posent problème sur l'efficacité de discrimination du groupe fondamental. Ce n'est pas une méthode efficace pour savoir si deux espaces sont homéomorphes, ou simplement homotopiquement équivalent, ou encore obtenu par déformation par rétraction. Un exemple facile serait le disque qui se rétracte par déformation en un point : les deux espace sont simplement connexes, mais ne sont pas homéomorphes.
\end{remark}
\end{comment}