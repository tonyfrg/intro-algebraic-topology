\section{Notations}
Pour une question de lisibilité et de clarté, nous avons choisi d'utiliser des notations classiques de topologie algébrique.
\begin{itemize}
    \item $\bb{Z}$ : le groupe des entiers relatifs, muni de l'addition ;
    \item $\bb{Z}_n$ : le groupe des entiers modulo $n$ (aussi noté $\bb{Z}/n\bb{Z}$) ;
    \item $\bb{R}^n$ : l'espace euclidien de dimension $n$ ;
    \item $D^n$ : le disque unité dans $\bb{R}^n$, l'ensemble de tous les vecteurs de longueur inférieure ou égale à~1 ;
    \item $\s{n}$ : la sphère unité dans $\bb{R}^{n+1}$, l'ensemble de tous les vecteurs de longueur 1 ;
    \item $\mathcal{V}(x)$ : l'ensemble des voisinages du point $x$ ;
    \item $\cong$ : la relation d'isomorphisme  ;
    \item $\simeq$ : l'équivalence d'homotopie ;
    \item $\sim$ : l'équivalence d'homologie ;
    \item $\approx$ : sauf mention contraire, pour une relation d'équivalence ;
    \item Pour deux applications $f,g$ telles que $dom(f)=cod(g)$, nous noterons $fg$ la composition de $f$ et $g$, habituellement notée $f\circ g$.
\end{itemize}

Sauf mention contraire, toutes les applications considérées seront supposées continues. Également, les espaces considérés seront supposés muni d'une topologie.