%delta complexe, homologie simpliciale 

Comme il est annoncé dans le titre, nous allons enfin pouvoir manipuler seulement des groupes abéliens. C'est le plus gros avantage à travailler sur l'homologie plutôt que sur le groupe fondamental. Un autre point fort de cet outil et que les groupes de dimensions supérieurs à l'espace sont triviaux, ce qui n'est pas le cas des groupes d'homotopies (voir le tableau \ref{fig:homotopy-groups-spheres}). En revanche, ce confort de manipulation à un coût : une plus forte abstraction des outils utilisés.

L'homologie peut être perçue comme étant l'abélianisation du groupe fondamental (voir section~\ref{sect:abelianisation} pour la démonstration), mais qu'est ce que cela représente concrètement ? Heureusement, la simplicité des calculs viendra contrebalancer les difficultés que nous pourrions avoir à appréhender le concept profond de l'homologie.

\bigskip Nous allons le voir rapidement, il existe différentes homologies, définies de façons différentes. Après les avoir introduits, ainsi que de voir leurs avantages et défauts, nous verrons qu'elles sont finalement isomorphes. Afin de mieux appréhender ces notions, nous allons commencer par l'homologie qui est la plus géométrique : l'homologie simpliciale.

\section{Homologie simpliciale : triangularisation}

L'homologie simpliciale ce calcule après avoir effectué une triangularisation de l'espace. Nous allons définir formellement comment obtenir une telle structure, en commençant par une généralisation des triangles.

\subsection{Des simplexes et des $\Delta$-complexes}\label{sect:simplices}
Intuitivement, un \emph{simplexe} est une généralisation du triangle, mais à n'importe quelle dimension. En dimension 1 il s'agit d'un segment, en dimension 2 un triangle, et en dimension 3 un tétrahèdre. Formellement, voici comme il est défini.

\begin{wrapfigure}{l}{0.3\textwidth}
\centering
\begin{tikzpicture}
\begin{scope}[thick,decoration={markings, mark=at position 0.5 with {\arrow{stealth}}}] 
    \draw[postaction={decorate}] (1,4.5)--(3,5.5);
    %2-simplex
    \draw[postaction={decorate}] (-.5,3)--(1,3);
    \draw[postaction={decorate}] (1,3)--(.5,4);
    \draw[postaction={decorate}] (-.5,3)--(.5,4);
    %3-simplex
    \draw[postaction={decorate}] (1.5,3)--(2.5,2.5);
    \draw[postaction={decorate}] (1.5,3)--(3.3,3.2);
    \draw[postaction={decorate}] (1.5,3)--(2.5,4);
    \draw[postaction={decorate}] (2.5,2.5)--(3.3,3.2);
    \draw[postaction={decorate}] (2.5,2.5)--(2.5,4);
    \draw[postaction={decorate}] (3.3,3.2)--(2.5,4);
\end{scope}
%0-simplex
\filldraw[] (0,5) circle(1pt) node[below]{$v_0$};
%1-simplex
\filldraw[] (1,4.5) circle(1pt) node[below]{$v_0$};
\filldraw[] (3,5.5) circle(1pt) node[below]{$v_1$};
%2-simplex
\filldraw[] (-.5,3) circle(1pt) node[below left]{$v_0$};
\filldraw[] (1,3) circle(1pt) node[below]{$v_1$};
\filldraw[] (.5,4) circle(1pt) node[above left]{$v_2$};
%3-simplex
\filldraw[] (1.5,3) circle(1pt) node[below]{$v_0$};
\filldraw[] (2.5,2.5) circle(1pt) node[below]{$v_1$};
\filldraw[] (3.3,3.2) circle(1pt) node[below]{$v_2$};
\filldraw[] (2.5,4) circle(1pt) node[above right]{$v_3$};

\end{tikzpicture}
\caption{\centering Simplexes de dimensions 0 à 3}
\label{tkz:simplex}
\end{wrapfigure}
\phantom{cyrille le goat}

\begin{definition}
Étant donné $n+1$ points~$v_0,...,v_n\in\bb{R}^m$ tels que les vecteurs~$v_i-v_0$ soient linéairement indépendant, un \emph{$n$-simplexe} est le plus petit ensemble convexe contenant tout ces points, que l'on note $[v_0,...,v_n]$. Les points $v_i$ sont appelés les \emph{sommets} du simplexe, et les sous-simplexes obtenu par le retrait d'un ou plusieurs sommets sont appelés les \emph{faces} du simplexe.
\end{definition}

\begin{remark}
Les simplexes possèdent des faces orientés. Par convention, l'ordre indiqué dans le nom de simplexe indique le sens des flèches. Par exemple, $[v_0,v_1]$ est un segment orienté de $v_0$ et $v_1$.
\end{remark}

\begin{exemple}
On définit le \emph{simplexe standard} par : \[\Delta^n=\left\{(t_0,...,t_n)\in\bb{R}^{n+1},\sum_{i=0}^nt_i=1,t_i\geq0,\forall i=0,...,n\right\}\]
\end{exemple}

\begin{definition}
Un $\Delta$-simplexe est un espace obtenu par une union de simplexes de différentes dimensions, auquel on identifie des faces via un homéomorphisme laissant invariant l'ordre les côtés.
\end{definition}

\begin{exemple}
Voici une représentation du tore en deux dimensions, du plan projectif réel, et de la bouteille de Klein, en tant que $\Delta$-complexes. Attention, ce ne sont pas les seules décompositions qui existent !

\begin{figure}[H]
\centering
\begin{subfigure}[t]{0.3\textwidth}
\centering
\begin{tikzpicture}[scale=.6]
% fond gris
\fill[gray!30] (0,0) rectangle (4,4);
\begin{scope}[thick,decoration={markings, mark=at position 0.5 with {\arrow{stealth}}}] 
    \draw[postaction={decorate}, red] (0,4)--(4,4)node[midway, below] {$a$};
    \draw[postaction={decorate}, red] (0,0)--(4,0) node[midway, above] {$a$};
    \draw[postaction={decorate}, blue] (0,0)--(0,4) node[midway, right] {$b$};
    \draw[postaction={decorate}, blue] (4,0)--(4,4) node[midway, left] {$b$};
    \draw[postaction={decorate}, brown] (4,0) -- (0,4)node[midway, below]{$c$};
\end{scope}
    \filldraw[] (4,0) circle(1pt) node[right]{$x_0$};
    \filldraw[] (0,4) circle(1pt) node[left]{$x_0$};
    \filldraw[] (4,4) circle(1pt) node[right]{$x_0$};
    \filldraw[] (0,0) circle(1pt) node[left]{$x_0$};
    \filldraw[gray] (1,1) circle(0pt) node[]{$\sigma_1$};
    \filldraw[gray] (3,3) circle(0pt) node[]{$\sigma_2$};
\end{tikzpicture}
\caption{Tore de dimension 2}
\label{tkz:torus}
\end{subfigure}
\begin{subfigure}[t]{0.3\textwidth}
\centering
\begin{tikzpicture}[scale=0.6]
    % fond gris
    \fill[gray!30] (0,0) rectangle (4,4);
\begin{scope}[thick,decoration={markings, mark=at position 0.5 with {\arrow{stealth}}}] 
    \draw[postaction={decorate}, red] (0,4)--(4,4)node[midway, below] {$a$};
    \draw[postaction={decorate}, red] (4,0)--(0,0) node[midway, above] {$a$};
    \draw[postaction={decorate}, blue] (0,0)--(0,4) node[midway, right] {$b$};
    \draw[postaction={decorate}, blue] (4,4)--(4,0) node[midway, left] {$b$};
    \draw[postaction={decorate}, brown] (4,0) -- (0,4)node[midway, below]{$c$};
\end{scope}
    \filldraw[] (4,0) circle(1pt) node[right]{$x_0$};
    \filldraw[] (0,4) circle(1pt) node[left]{$x_0$};
    \filldraw[] (4,4) circle(1pt) node[right]{$x_0$};
    \filldraw[] (0,0) circle(1pt) node[left]{$x_0$};
    \filldraw[gray] (1,1) circle(0pt) node[]{$\sigma_1$};
    \filldraw[gray] (3,3) circle(0pt) node[]{$\sigma_2$};
\end{tikzpicture}
\caption{Plan projectif}
\label{tkz:proj-plane-simp}
\end{subfigure}
\begin{subfigure}[t]{0.3\textwidth}
\centering
\begin{tikzpicture}[scale=.6]
    % fond gris
    \fill[gray!30] (0,0) rectangle (4,4);
    %flèche du haut
\begin{scope}[thick,decoration={markings, mark=at position 0.5 with {\arrow{stealth}}}] 
    \draw[postaction={decorate}, red] (0,4)--(4,4)node[midway, below] {$a$};
    \draw[postaction={decorate}, red] (4,0)--(0,0) node[midway, above] {$a$};
    \draw[postaction={decorate}, blue] (0,0)--(0,4) node[midway, right] {$b$};
    \draw[postaction={decorate}, blue] (4,0)--(4,4) node[midway, left] {$b$};
    \draw[postaction={decorate}, brown] (4,0) -- (0,4)node[midway, below]{$c$};
\end{scope}
    \filldraw[] (4,0) circle(1pt) node[right]{$x_0$};
    \filldraw[] (0,4) circle(1pt) node[left]{$x_0$};
    \filldraw[] (4,4) circle(1pt) node[right]{$x_0$};
    \filldraw[] (0,0) circle(1pt) node[left]{$x_0$};
    \filldraw[gray] (1,1) circle(0pt) node[]{$\sigma_1$};
    \filldraw[gray] (3,3) circle(0pt) node[]{$\sigma_2$};
\end{tikzpicture}
\caption{Bouteille de Klein}
\label{tkz:klein-bottle}
\end{subfigure}
\end{figure}
\end{exemple}

\subsubsection{L'homologie}
Lorsque l'on a un $\Delta$-complexe $X$, nous avons aussi une structure qui a été choisi, la manière dont nous avons recouvert l'espace de simplexes. Avec ce choix, nous avons une liste de simplexes, que l'on peut trier par rapport à la dimension.

Ainsi, nous pouvons nous intéresser au groupe libre abélien formé par les combinaisons de simplexes de même dimension. C'est ce que l'on note $\Delta_n(X)$, et l'on appellera les éléments des \emph{n-chaînes}, écrit sous la forme de sommes de~$n$-simplexes.

\bigskip Il est courant d'utiliser la notation $e_i^n$ pour un $n$-simplexe de dimension $n$, de telle sorte que l'on définisse les $n$-chaînes s'écrivant $\sum_in_ie_i^n$, avec $n_i\in\bb{Z}$. Mais nous préférerons ici utiliser la notation $\sum_in_i\sigma_i$, avec $\sigma:\Delta^n\to X$ \emph{l'application caractéristique} de $e_i^n$. Ce choix se justifie d'une part pour correspondre à la notation choisie dans \cite{Hatcher}, mais de plus car elle sera reprise pour l'homologie singulière.

\begin{exemple}
Pour un $2$-simplexe $X=[v_0,v_1,v_2]$, nous avons :$$\Delta_1(X)=\{\alpha_0[v_1,v_2]+\alpha_1[v_0,v_2]+\alpha_2[v_0,v_1],\,\alpha_0,\alpha_1,\alpha_2\in\bb{Z}\}.$$
\end{exemple}

Nous pouvons remarquer dans un premier temps que les faces d'un simplexe sont des simplexes. C'est à dire qu'il existe un lien entre les chaînes d'une certaine dimension et les chaînes de dimension inférieure. Une relation qui s'apparente bien est celle que l'on appelle la \emph{frontière}, ou la \emph{bordure} (boundary en anglais). L'idée est de passer d'un $n$-simplexe à une somme de $n-1$-simplexes, correspond à un tour effectué le bord de ce simplexe. Voici une définition plus formelle :

\begin{definition}
L'application bordure $\partial:[v_0,...,v_n]\mapsto\sum_i(-1)^i[v_0,...,\hat{v_i},...,v_n]$, où $\hat{v_i}$ désigne que l'on omet le sommet $v_i$ du simplexe, s'étend par linéarité sur les $n$-chaînes, de telle sorte que l'on puisse définir un \emph{morphisme bordure} sur les groupes de chaînes $\partial_n:\Delta_n(X)\to\Delta_{n-1}(X)$, que l'on définie sur les éléments de la base de $\Delta_n(X)$ par : \[\partial_n(\sigma)=\sum_i(-1)^i\sigma|_{[v_0,...,\hat{v_i},...,v_n]}.\]
\end{definition}

\begin{proposition}
Pour $n\geq0$, $\partial_{n-1}\partial_n=0$.
\end{proposition}
\begin{proof}
La preuve se fait par un simple calcul. Pour $\sigma=[v_0,...v_n]$, nous avons : \[\begin{split}
\partial_{n-1}\partial_n(\sigma)=&\sum_{j<i}(-1)^i(-1)^j\sigma|_{[v_0,...,\hat{v_i},...,\hat{v_j},...,v_n]}\\
&+\sum_{j>i}(-1)^i(-1)^{j+1}\sigma|_{[v_0,...,\hat{v_j},...,\hat{v_i},...,v_n]}.
\end{split}\]Les deux sommes sont égales, au signe près. On en conclut que cela fait 0. 
\end{proof}


\subsection{Homologie}\label{sect:simpl-homology}

Si l'on possède une famille de groupes abéliens vérifiant cette propriété, nous avons alors une séquence de morphisme, comme le diagramme suivant, que l'on appelle \emph{complexe de chaînes} :

\[\begin{tikzcd}
\cdots\arrow[r]& C_{n+1}\arrow[r, "\partial_{n+1}"]&C_n\arrow[r,"\partial_n"]&C_{n-1}\arrow[r] & \cdots,
\end{tikzcd}\]
Si en plus nous avons $\partial_n\partial_{n+1}=0$, pour tout $n$, et $\partial_0=0$, on obtient la relation~$\im(\partial_{n+1})\subset\ker(\partial_n)$, permettant ainsi de passer au quotient, et donc de définir ce qu'est qu'une homologie.

\begin{definition}\label{def:homology}
Le $n$-ième \emph{groupe d'homologie} d'un complexe de chaînes est défini comme étant le groupe quotient $H_n=\ker\partial_n/\im\partial_{n+1}$. Les éléments de~$\ker\partial_n$ sont appelés les \emph{cycles} et les éléments de $\partial_{n+1}$ sont les \emph{bordures}, et les éléments de $H_n$ sont les \emph{classes d'homologies}. Deux cycles qui représentent la même classe d'homologie sont dit \emph{homologues}.
\end{definition}

Revenons maintenant aux $\Delta$-complexes. C'est à dire que l'on a $C_n=\Delta_n(X)$, et notre groupe d'homologie sera noté $\homsimp_n(X)$, pour le $n$-ième groupe d'homologie simpliciale.

\bigskip Voyons désormais comment utiliser l'homologie en pratique.


\subsubsection{Application de l'homologie simpliciale}

\begin{exemple}
Soit $X=\s{1}$. Supposons que l'on construise l'espace avec un $1$-simplexe $e=[v,w]$, où l'on procède à l'identification $[v]=[w]$. Nous avons alors le complexe de chaînes simpliciales suivant :
\[\begin{tikzcd}
0\arrow[r,"\partial_2"]& \langle e\rangle\arrow[r, "\partial_1"]&\langle v\rangle\arrow[r,"\partial_0"]&0.
\end{tikzcd}\]
Il est évident que $\partial_2=0$ et $\partial_0=0$. Pour définir le morphisme $\partial_1$, il suffit de voir l'image par $e$. Or $\partial_1(e)=v-v=0$. On en déduit alors que $\ker\partial_1=\langle e\rangle$ et $\im\partial_1=0$. Ce qui nous donne les groupes d'homologies simpliciales suivants : 
\[\begin{split}
    \homsimp_2(\s{1}) &= \ker\partial_2 = 0 \\
    \homsimp_1(\s{1}) &= \ker\partial_1/\im\partial_2 = \ker\partial_1 = \langle e \rangle\cong\bb{Z} \\
    \homsimp_0(\s{1}) &= \ker\partial_0/\im\partial_1 = \langle v \rangle/\langle 0 \rangle = \langle v \rangle\cong\bb{Z}.
\end{split}\]

Cette fois, nous construisons l'espace $X=\s{1}$, avec deux $1$-simplexes $e,f$, auquel nous avons identifié leurs sommets (et uniquement leur sommet. Nous nous retrouvons alors avec deux 0-simplexes $v_0,v_1$, ce qui nous donne le complexe de chaînes suivant : 
\[\begin{tikzcd}
0\arrow[r,"\partial_2"]& \langle e,f\rangle\arrow[r, "\partial_1"]&\langle v_0,v_1\rangle\arrow[r,"\partial_0"]&0.
\end{tikzcd}\]
De même que pour le cas précédent, nous avons $\partial_2=0$ et $\partial_0=0$. Ensuite, nous avons $\partial_1(e)=v_1-v_0$ et $\partial_1(f)=v_1-v_0$. Nous pouvons en déduire donc que $\partial_1(e-f)=0$, et ainsi que $\ker\partial_1=\langle e-f\rangle$. Nous pouvons dès lors calculer les groupes d'homologies : \[\begin{split}
\homsimp_2(\s{1}) &= 0 \\
\homsimp_1(\s{1}) &= \ker\partial_1 = \langle e-f \rangle \cong\bb{Z}\\
\homsimp_0(\s{1}) &= \ker\partial_0/\im\partial_1 = \langle v_0,v_1 \rangle/\langle v_1-v_0 \rangle = \langle v_0,v_1|v_1-v_0=0 \rangle\cong\bb{Z}.
\end{split}\]
\end{exemple}

L'homologie simpliciale nous permet de calculer les groupes d'homologies d'espaces simples, tels que le tore en deux dimensions, le ruban de Möbius, le plan projectif réel, la sphère $\s{2}$, etc. Vous pouvez retrouver ces calculs dans l'annexe \ref{chap:annexe-simp-homo}.

\begin{remark}
Nous pouvons remarquer dans l'exemple ci-dessus que les deux structures de $\Delta$-complexes choisies de $\s{1}$ nous donnent les mêmes groupes d'homologies simpliciales. Néanmoins, ce n'est pas évident, au vu de la définition, que le groupe d'homologie simpliciale ne dépende pas de la structure de $\Delta$-complexe choisie. Il s'avère que c'est vrai, mais parce que l'homologie simpliciale est isomorphe à l'homologie singulière, mais sans ce résultat, nous ne pourrions pas en être certain.
\end{remark}

\section{Homologie singulière : la théorie}

\begin{definition}\label{def:homology-sing}
Un \emph{n-simplexe singulier} est une application $\sigma:\Delta^n\to X$ continue (le terme singulier vient du fait que l'on autorise l'application à avoir des "singularités"). Nous notons~$C_n(X)$ le groupe abélien libre ayant comme base l'ensemble des $n$-simplexes singulier sur l'espace $X$. 

De même que pour les simplexes simpliciaux, nous appellerons \emph{n-chaîne singulier} ou $n$-chaîne un élément de~$C_n(X)$, qui s'écrit sous la forme $\sum_in_i\sigma_i$, avec~$\sigma_i$ un $n$-simplexe singulier. Nous pouvons également définir une application bordure $\partial_n:C_n(X)\to C_{n-1}(X)$ sur les $n$-chaînes, par : \[\partial_n(\sigma)=\sum_i(-1)^i\sigma|_{[v_0,\cdots,\hat{v_i},\cdots,v_n]},\] où l'on identifie  $[v_0,\cdots,\hat{v_i},\cdots,v_n]$ comme étant $\Delta^{n-1}$, tout en préservant l'ordre des sommets.

De même que pour l'homologie simpliciale, nous avons $\partial_n\partial_{n+1}=0$, si bien que l'on peut définir l'\emph{homologie singulière} par $H_n(X)=\ker\partial_n/\im\partial_{n+1}$.
\end{definition}

Nous avons fini la partie sur l'homologie simpliciale en remarquant le fait que l'on était pas certain d'obtenir une unicité du groupe d'homologie, notamment en changeant la structure de $\Delta$-complexe associée. Comme nous pouvons le voir avec la définition, nous n'avons besoin d'aucune structure au préalable pour le calcul de l'homologie : elle ne dépend seulement que de l'espace en question. En particulier, elle est clairement invariante par homéomorphisme, ce qui n'est pas évident avec l'homologie simpliciale.

Un point majeur à soulever toutefois se trouve sur les groupes de chaînes singulières. Sur la plupart des espaces il existe un nombre incalculable d'applications continues, et il en est de même des simplexes singuliers. Ce qui fait que l'on se trouve avec des groupes de chaînes singulières gigantesque, dont il serait interminable le calcul des groupes d'homologies. Mais c'est normal ! En réalité, l'homologie singulière n'est pas faite pour les calculs, elle est simplement faite pour les démonstration, car il est facile de la manipulée pour démontrer des résultats théoriques.

\subsection{Premières propriétés sur l'homologie singulière}

\begin{proposition}
Soit $X$ un espace, avec comme décomposition en composante connexe par arc~$(X_i)_{i\in I}$. Alors $H_n(X)\cong\oplus_{i\in I} H_n(X_i)$.
\end{proposition}
\begin{proof}
Comme les simplexes singuliers ont leurs images connexes par arc, nous pouvons définir~$C_n(X)$ comme somme directe des $C_n(X_i)$. De plus, la somme directe est conservée par le morphisme de bordure puisqu'elle envoie les éléments de chaque $C_n(X_i)$ vers $C_{n-1}(X_{i})$. Il s'en suit alors que~$\ker\partial_n$ et $\im\partial_{n+1}$ se décompose en somme directe, de telle sorte qu'il en est de même pour~$H_n(X)$.
\end{proof}

\begin{proposition}
Si $X$ est non vide et connexe par arc, alors $H_0(X)\cong\bb{Z}$.
\end{proposition}
\begin{proof}
Du fait que $\partial_0=0$, nous avons $H_0(X)=C_0(X)/\im\partial_1$. En définissant le morphisme~${\varepsilon:C_0(X)\to \bb{Z}}$ par $\varepsilon(\sum_in_i\sigma_i))=\sum_in_i$, nous allons montrer qu'il induit un isomorphisme entre~$H_0(X)$ et~$\bb{Z}$, autrement dit que l'on a $\ker\varepsilon=\im\partial_1$.

Premièrement, il est clair que le morphisme est surjectif lorsque $X\neq\emptyset$ : pour $n\in \bb{Z}$, nous pouvons associer un 1-simplexe singulier $\sigma$ tel que $\varepsilon(n\sigma)=n$. De plus, nous avons $\im\partial_1\subset\ker\varepsilon$, car pour un 1-simplexe singulier $\sigma$, nous avons : $$\varepsilon\partial_1(\sigma)=\varepsilon(\sigma|_{[v_1]}-\sigma|_{[v_0]})=1-1=0.$$
Il reste à démontrer que $\ker\varepsilon\subset\im\partial_1$. On commence par définir $\sigma=\sum_in_i\sigma_i$ tel que $\sum_in_i=0$, ie.~$\sigma\in\ker\varepsilon$. En voyant les $\sigma_i$ comme des points, nous pouvons définir un chemin entre eux. Notons alors~$\tau_i$ le chemin partant du $\sigma_0$ vers $\sigma_i$. Ces $\tau_i$ peuvent être perçus comme des 1-simplexes singuliers\footnote{Nous pouvons observer le lien entre le groupe fondamental et le premier groupe d'homologie.}, étant donné que $\Delta^1$ est homéomorphe à~$[0,1]$, vérifiant~$\partial_1\tau_i=\sigma_i-\sigma_0$. Nous obtenons alors le résultat suivant : \[\partial_1\left(\sum_in_i\tau_i\right)=\sum_in_i(\sigma_i-\sigma_0)=\sum_in_i\sigma_i-\sum_in_i\sigma_0=\sum_in_i\sigma_i=\sigma,\]du fait que $\sum_in_i=0$. Nous en déduisons donc que $\sigma\in\im\partial_1$. Nous en concluons que~${\ker\varepsilon\subset\im\partial_1}$.
\end{proof}

\begin{proposition}
Pour $X$ un point, nous avons $H_0(X)=\bb{Z}$ et $H_n(X)=0$ pour $n>0$.
\end{proposition}
\begin{proof}
Comme $X$ n'est qu'un élément, les simplexes singuliers sont uniques à chaque dimension $n$, notons les $\sigma_n$, et sont constants. Pour $\sigma_n\in C_n(X)$, nous avons $\partial_n(\sigma)=\sum_i(-1)^i\sigma_{n-1}$, une suite de $n+1$ termes. Il s'en suit alors le complexe de chaînes suivant : 
\[\begin{tikzcd}
\centering
\cdots\arrow[r]&\bb{Z}\arrow[r,"\cong"]&\bb{Z}\arrow[r,"0"]&\bb{Z}\arrow[r,"\cong"]&\bb{Z}\arrow[r,"0"]&\bb{Z}\arrow[r]&0.
\end{tikzcd}\]
Il s'en suit alors que nous avons $H_0(X)=\ker\partial_0\cong\bb{Z}$, et pour~$n>0$, nous avons~$\im\partial_{n+1}=\ker\partial_n$. Les groupes d'homologies sont donc triviaux pour~$n>~0$.
\end{proof}

\begin{remark}
Étant donné le résultat que nous venons de montrer, il est parfois préférable de travailler avec des groupes d'homologies vérifiant que celle du point vaut toujours 0. Nous définissons alors l'homologie réduite $\Tilde{H}_n(X)$ comme étant égale à $H_n(X)$ pour $n>1$, et $H_0(X)=\Tilde{H}_0(X)\oplus\bb{Z}$.
\end{remark}

\subsection{Invariance par homotopie}

Dans une quête d'outils plus intéressant que le groupe fondamental, nous aimerions au moins qu'il vérifie les mêmes "bonnes"propriétés que celui-ci. Parmi elles se trouve la fonctorialité, permettant d'induire des morphismes sur les groupes d'homologie, à partir d'une application entre deux espaces.

Mais avant de voir le comportement des applications sur les groupes d'homologies, nous allons voir comment cela ce passe sur les groupes  de chaînes.

\begin{definition}\label{def:chain-map}
Soit $f:X\to Y$ une application. Le morphisme induit sur les groupes de chaînes~$f_\#:C_n(X)\to C_n(Y)$ est défini par la composition de chaque $n$-simplexes singulier~$\sigma:\Delta^n\to X$ avec $f$, afin d'obtenir un $n$-simplexe singulier de $f_\#(\sigma)=f\sigma:\Delta^n\to Y$, et en étendant linéairement pour obtenir : \[f_\#\left(\sum_in_i\sigma_i\right)=\sum_in_if_\#(\sigma_i)=\sum_in_if\sigma_i.\]On appelle $f_\#$ une \emph{application de chaîne}.
\end{definition}

Avec les mêmes notations, nous avons la propriété $\partial f_\#=f_\#\partial$, du fait que l'on étende linéairement l'application $f$. Nous obtenons ainsi le \emph{diagramme commutatif} suivant :

\[\begin{tikzcd}
\cdots\arrow[r]&C_{n+1}(X)\arrow[r,"\partial"]\arrow[d,"f_\#"]&C_n(X)\arrow[r,"\partial"]\arrow[d,"f_\#"]&C_{n-1}(X)\arrow[r,"\partial"]\arrow[d,"f_\#"]&\cdots\\
\cdots\arrow[r]&C_{n+1}(Y)\arrow[r,"\partial"]&C_n(Y)\arrow[r,"\partial"]&C_{n-1}(Y)\arrow[r,"\partial"]&\cdots\\
\end{tikzcd}\]

\begin{proposition}
Une application de chaîne $f_\#:C_n(X)\to C_n(Y)$ entre deux complexes de chaînes induisent des morphismes sur les groupes d'homologies $f_\ast:H_n(X)\to H_n(Y)$.
\end{proposition}
\begin{proof}
Soit $f_\#$ une application de chaîne. Puisqu'elle commute avec l'application $\partial$, les cycles de $X$ sont également des cycles de $Y$ : si $\partial\alpha=0$, alors~$\partial (f_\#\alpha)=f_\#(\partial\alpha)=0$. De même, les bordures de $X$ sont les bordures de~$Y$ : $f_\#(\partial\beta)=\partial f_\#(\partial\beta)$. Alors, $f_\#$ induit un morphisme entre les groupes d'homologies.
\end{proof}

Il est assez facile de vérifier par la suite que l'application induite sur les groupes d'homologies corresponde bien à un foncteur entre espaces et groupes abéliens : \begin{itemize}
    \item Nous avons $id_\ast=id$ ;
    \item Nous avons $(fg)_\ast=f_\ast g_\ast$, pour $fg:X\to Y\to Z$, du fait qu'il s'agisse simplement d'une composition \begin{tikzcd}
    \Delta^n\arrow[r,"\sigma"]&X\arrow[r,"g"]&Y\arrow[r,"f"]&Z.
    \end{tikzcd}
\end{itemize}

Avec ce morphisme induit, nous retrouvons quelques bonne propriétés, qui se résument en le théorème suivant.

\begin{theorem}
Si deux applications $f,g:X\to Y$ sont homotopes, alors elles induisent le même morphisme de groupes $f_\ast=g_\ast:H_n(X)\to H_n(Y)$.
\end{theorem}
\begin{proof}
Nous devons montrer que $g_\ast\sim f_\ast$. Pour cela, on introduit l'idée de voir $\Delta^n\times[0,1]$ comme un $\Delta$-complexe, formé de $(n+1)$-simplexes. Ensuite, en notant $F:X\times I\to Y$ l'homotopie entre $f$ et $g$, nous introduisons une application $P$, appelé \emph{opérateur prisme}, définie par : \[P(\sigma)=\sum_i (-1)^iF(\sigma\times i d)-_[v_0,...,v_i,w_i,...,w_n].\]L'application est définie pour vérifier la propriété $\partial P+P\partial = g_\#-f_\#$, de telle sorte que l'on puisse conclure avec le raisonnement suivant. Pour $\alpha\in C_n(X)$ un cycle, c'est à dire $\partial\alpha=0$ nous avons~${g_\#(\alpha)-f_\#(\alpha)=\partial P(\alpha)+P(\partial\alpha)=\partial P(\alpha)}$. On en déduit donc que~$g_\#-f_\#\in\im\partial$, ce qui signifie que les classes d'homologies $f_\#(\alpha)$ et $g_\#(\alpha)$ sont les mêmes, et donc que $f_\ast$ et $g_\ast$ sont égales pour la classe d'homologie de~$\alpha$.
\end{proof}

Pour une démonstration plus complète, le lecteur peut se diriger vers \cite{Hatcher}, page 112.

Voici quelques applications directes du théorème.

\begin{corollary}
Une équivalence d'homotopie induit des isomorphismes sur les groupes d'homologies, pour tout $n$.
\end{corollary}
\begin{exemple}
Si $X$ est contractible, alors $H_n(X)\cong H_n(point)$.
\end{exemple}

\subsection{Séquences exactes et homologie relative}

Il serait intéressant de trouver un lien entre les homologies d'un espace $X$, d'un sous-espace $A$, et encore d'un espace quotient $X/A$. Avec ce que l'on vient de voir, nous pouvons imaginer que l'inclusion $A\hookrightarrow X$ induise une injection canonique. De même, le quotient $q:X\to X/A$ pourrait induire une surjection sur les groupes d'homologies. Mais comment faire tenir tout ceci sur une seule chaîne ?

L'homologie relative est la solution. Souvent comparée à l'arithmétique modulaire pour les espaces, car on peut l'appréhender comme étant un modulo sur les homologies, l'homologie relative est une simplification des espaces permettant de démontrer de nombreux résultats, comme nous allons le voir par la suite.

\subsubsection{Homologie relative}
\begin{definition}
Soit $X$ un espace et $A\subset X$ un sous-espace. On note $C_n(X,A)$ le groupe quotient~$C_n(X)/C_n(A)$, rendant les chaînes de $A$ triviales. L'application bordure induit une application sur le quotient $\partial:C_n(X,A)\to C_{n-1}(X,A)$, avec laquelle nous obtenons le \emph{complexe de chaînes relatives}. La relation $\partial^2=0$ tient toujours, nous pouvons alors définir \emph{l'homologie de X relativement à A}, noté $H_n(X,A)$, par : \begin{itemize}
    \item Les éléments de $H_n(X,A)$ sont nommés les \emph{cycles relatifs} : $n$-chaînes $\alpha\in C_n(X)$ tel que~${\partial\alpha\in C_n(A)}$ ;
    \item Un cycle relatif est trivial dans $H_n(X,A)$ si et seulement s'il s'agit d'une \emph{bordure relative} défini par $\alpha=\partial\beta+\upgamma$, avec $\beta\in C_{n+1}(X)$ et $\upgamma\in C_n(A)$.
\end{itemize}
\end{definition}

Notre objectif, avec l'homologie relative, est de construire une séquence telle que la suivante, pour $X$ un espace et $A\subset X$ : \[\begin{tikzcd}
\cdots\arrow[r]&H_n(A)\arrow[r]&H_n(X)\arrow[r]&H_n(X,A)\arrow[r]&H_{n-1}(A)\arrow[r]&\cdots,
\end{tikzcd}\]et pour cela, nous devons prendre du recul et voir un peut de théorie sur les séquences.

\subsubsection{Séquences exactes}

\begin{definition}
Une séquence telle que la suivante et dites \emph{exacte} lorsque pour tout $n$, elle vérifie~${\ker\alpha_{n}=\im\alpha_{n+1}}$. \[\begin{tikzcd}
\cdots\arrow[r]&A_{n+1}\arrow[r, "\alpha_{n+1}"]&A_n\arrow[r, "\alpha_n"]&A_{n-1}\arrow[r]&\cdots
\end{tikzcd}\]
\end{definition}

\begin{exemple}
Voici quelques exemples de conditions pour qu'une séquence soit exacte :
\begin{enumerate}
\item \begin{tikzcd}
0\arrow[r]&A\arrow[r, "\alpha"]&B
\end{tikzcd} est une séquence exacte ssi $\alpha$ est injective.
\item \begin{tikzcd}
A\arrow[r, "\alpha"]&B\arrow[r]&0
\end{tikzcd} est une séquence exacte ssi $\alpha$ est surjective.
\item \begin{tikzcd}
0\arrow[r]&A\arrow[r, "\alpha"]&B\arrow[r]&0
\end{tikzcd} est une séquence exacte ssi $\alpha$ est un isomorphisme.
\item \begin{tikzcd}
0\arrow[r]&A\arrow[r, "\alpha"]&B\arrow[r, "\beta"]&C\arrow[r]&0
\end{tikzcd} est une séquence exacte ssi $\alpha$ est injective, $\beta$ surjective, et que $\ker\beta=\im\alpha$ : c'est à dire si $\beta$ induit un isomorphisme $C\cong B/\im\alpha$. On appelle ceci une \emph{séquence exacte courte}.
\end{enumerate}
\end{exemple}

Avec les complexes de chaînes, nous pouvons construire des séquences exactes courtes. De plus, l'application bordure permet d'obtenir une séquence de séquences exactes courtes : 
\[
\footnotesize
\begin{tikzcd}
&0\arrow[d]&0\arrow[d]&0\arrow[d]&\\
\cdots\arrow[r]&C_{n+1}(A)\arrow[r, "\partial"]\arrow[d,"i"]&C_n(A)\arrow[r, "\partial"]\arrow[d,"i"]&C_{n-1}(A)\arrow[r]\arrow[d,"i"]&\cdots\\
\cdots\arrow[r]&C_{n+1}(X)\arrow[r, "\partial"]\arrow[d,"q"]&C_n(X)\arrow[r, "\partial"]\arrow[d,"q"]&C_{n-1}(X)\arrow[r]\arrow[d,"q"]&\cdots\\
\cdots\arrow[r]&C_{n+1}(X,A)\arrow[r, "\partial"]\arrow[d]&C_n(X,A)\arrow[r, "\partial"]\arrow[d]&C_{n-1}(X,A)\arrow[r]\arrow[d]&\cdots\\
&0&0&0&
\end{tikzcd}\]
Ce qu'il nous manque ici pour aboutir au résultat escompté, c'est une application permettant de passer du groupe $C_n(X,A)$ au groupe $C_{n-1}(A)$. Sur le diagramme, cela reviendrait à trouver une application qui nous permettrait de remonter.

\bigskip Pour être plus claire dans l'explication qui va suivre, nous allons voir que cela fonctionne dans le cas général, lorsque l'on est dans la même configuration de diagramme. Soit $A,B,C$ trois espaces ayant comme complexes de chaînes la suivante, ainsi que des séquences courtes exactes verticales : 
\[
\footnotesize
\begin{tikzcd}
&0\arrow[d]&0\arrow[d]&0\arrow[d]&\\
\cdots\arrow[r]&A_{n+1}\arrow[r, "\partial"]\arrow[d,"i"]&A_n\arrow[r, "\partial"]\arrow[d,"i"]&A_{n-1}\arrow[r]\arrow[d,"i"]&\cdots\\
\cdots\arrow[r]&B_{n+1}\arrow[r, "\partial"]\arrow[d,"j"]&B_n\arrow[r, "\partial"]\arrow[d,"j"]&B_{n-1}\arrow[r]\arrow[d,"j"]&\cdots\\
\cdots\arrow[r]&C_{n+1}\arrow[r, "\partial"]\arrow[d]&C_n\arrow[r, "\partial"]\arrow[d]&C_{n-1}\arrow[r]\arrow[d]&\cdots\\
&0&0&0&
\end{tikzcd}\]Notre objectif est donc de définir une application $\partial:H_n(C)\to H_{n-1}(A)$. Pour cela, on considère un cycle~$ c\in C_n$. Puisque $j$ est surjectif par l'exactitude de la séquence, il existe~$b\in B_n$ tel que~$j(b)=c$. Or, $\partial b\in B_{n-1}$ appartient à~$\ker j$, du simple fait que $j(\partial b)=\partial j(b)=\partial c=0$. Dès lors, comme~${\ker j=\im i}$, nous savons qu'il existe $a\in A_{n-1}$ tel que $i(a)=\partial b$, dont on pourra remarquer~$\partial a=0$, du fait que $i$ est injectif et $i(\partial a)=\partial i(a)=\partial\partial b=0$. Il s'agit alors d'un cycle de~$A_{n-1}$.

Nous définissons ainsi la bordure $\partial:H_n(C)\to H_{n-1}(A)$, qui à une classe d'homologie $[c]$ renvoie~$\partial[c]=[a]$. Cette application est bien définie du fait qu'elle ne dépend pas du représentant choisi. Un autre représentant de la classe de $c$ est de la forme $c+\partial c'$, avec le terme $\partial c'$ qui partira lorsque l'on composera l'élément de $B_n$ correspondant avec $\partial$.

\begin{theorem}
La séquence de groupes d'homologies : \[\footnotesize\begin{tikzcd}
\cdots\arrow[r]&H_n(A)\arrow[r,"i_\ast"]&H_n(B)\arrow[r,"j_\ast"]&H_n(C)\arrow[r,"\partial"]&H_{n-1}(A)\arrow[r,"i_\ast"]&H_{n-1}(B)\arrow[r]&\cdots,
\end{tikzcd}\]est exacte.
\end{theorem}
L'idée de la démonstration est de vérifier les inclusions suivantes \cite{Hatcher} : \[\im i_\ast\subset \ker j_\ast,\,\im j_\ast\subset\ker\partial,\,\im\partial\subset\ker i_\ast,\,\ker j_\ast\subset\im i_\ast,\,\ker\partial\subset\im j_\ast,\,\ker i_\ast\subset i_\ast.\]
\begin{remark}
Cette méthode de preuve est courante en algèbre homologique et en théorie des catégories, si bien qu'elle porte le nom de \emph{diagramme chasing}, venant de l'anglais \emph{diagram chasing}.
\end{remark}

Si l'on revient à nos espaces topologiques, nous venons de construire une séquences de groupes d'homologies relatives, un outils très puissant, comme nous pouvons le voir dans l'exemple suivant.

\begin{exemple}
Notre but est de calculer les groupes d'homologies de $\s{n}$. Pour le moment, nous n'avons pas tous les outils pour aboutir le calcul, mais nous pouvons commencer notre raisonnement. Soit~$D^n$ notre espace, et $\partial D^n=\s{n-1}\subset D^n$. Du fait que $D^n$ soit contractible, nous avons la séquence exacte suivante :\[\begin{tikzcd}
\Tilde{H}_k(D^n)=0\arrow[r]&\Tilde{H}_k(D^n,\partial D_n)\arrow[r,"\partial"]&\Tilde{H}_{k-1}(\s{n-1})\arrow[r]&0=\Tilde{H}_{k-1}(D^n),
\end{tikzcd}\]permettant de déduire que les homologies $\Tilde{H}_k(D^n,\partial D^n)$ et~$\Tilde{H}_{k-1}(\s{n-1})$ sont isomorphes. Sachant que $\s{n}$ est homéomorphe à $D^n/\partial D^{n-1}$, nous pouvons remarquer une certaine relation de récurrence apparaître...
\end{exemple}

\subsection{Excision}
Le théorème qui suit, probablement l'un des plus puissant, permet d'affirmer que l'homologie relative $H_n(X,A)$ ne s'intéresse pas de l'intérieur de $A$. Ce résultat nous permettra notamment de lier les groupes d'homologies d'espaces, et d'espaces quotients.

\begin{theorem}
Pour $X$ un espace, et soient $Z\subset A\subset X$ tel que l'adhérence de $Z$ soit inclus dans l'intérieur de $A$, nous avons~${H_n(X,A)\cong H_n(X\setminus Z,A\setminus Z)}$.

On en déduit directement que pour $X$ un espace, et $A,B\subset X$ tel que $X$ est un recouvrement de l'union des intérieurs de $A$ et $B$, l'inclusion $(B,A\cap B)\hookrightarrow(X,A)$ induit un isomorphisme entre~$H_n(X,A)$ et~$H_n(B, A\cap B)$.
\end{theorem}
\begin{remark}
Pour passer d'un résultat à l'autre, il suffit de prendre $Z=X\setminus B$.
\end{remark}
\begin{proof}
Premièrement, nous pouvons remarquer que pour $X$ avec une structure de $\Delta$-complexe, la composition $\Phi:\Delta_n(X\setminus Z)\to \Delta_n(X)\to \Delta_n(X,A)$ est surjective, car la base de~$\Delta_n(X,A)=\Delta_n(X)/\Delta_n(A)$ sont donnés par les éléments qui ne sont pas dans $A$, que l'on peut alors retrouver dans la base de~$\Delta_n(X\setminus Z)$. Son noyau est aussi réduit aux éléments de la base qui sont dans~$A$ mais pas dans $Z$, autrement dit $\ker\Phi=\Delta_n(A\setminus Z)$. Nous obtenons ainsi directement l'isomorphisme souhaité. 

En général, ce n'est pas aussi simple. Pour réussir à démontrer le résultat, il nous faut subdiviser les simplexes singuliers pour qu'ils soient présent soit dans~$A$, soit dans $X\setminus Z$ : ce qui donne un recouvrement de $X$, du fait que les simplexes singuliers soient compacts. Si l'on note $\mathcal{U}$ un tel recouvrement de $X$, on défini $C_n^\mathcal{U}(X)$ le groupe libre des $n$-simplexes singuliers appartenant à $\mathcal{U}$, et l'on montre que l'inclusion $C_n^\mathcal{U}(X)\hookrightarrow C_n(X)$ induit un isomorphisme sur les groupes d'homologies~${H_n^\mathcal{U}(X)\cong H_n(X)}$.

On démontre ce résultat en construisant deux applications sur les chaînes : ${S:C_n(X)\to C_n(X)}$ et~${T:C_n(X)\to C_{n+1}(X)}$, définie de telle sorte sorte qu'elle vérifie $\partial T+T\partial=id-S$, et qu'elles vérifient, pour $f:X\to Y$, $Sf_\#=f_\#S$ et $Tf_\#=f_\#T$. Une dernière propriété serait que pour tout~$\sigma:\Delta^n\to X$, il existe $m\in\bb{N}$ tel que $S^m\sigma\in C_n^\mathcal{U}(X)$. En notant $T_m=T(id+S+\cdots+S^{m-1})$, cela implique d'avoir~$\partial T_m+T_m\partial=id-S^m$ Le lecteur voulons savoir comment sont définies de telles applications peut voir \cite{Hatcher} page 120 à 124, ici, nous nous contenterons d'expliquer en quoi elles sont utiles.

\bigskip Supposons qu'il existe de tels applications $S$ et $T$, montrons que l'inclusion induise un isomorphisme $i_\ast$ entre $H_n^\mathcal{U}(X)$ et $H_n(X)$. Soit $[c]\in H_n(X)$, c'est à dire~$c\in C_n(X)$ tel que $\partial c=0$. Il existe $m$ tel que $S^mc\in C_n^\mathcal{U}(X)$. Or, on a les deux résultats : $$\partial T_mc+T_m\partial c=\partial T_mc=c-S^mc,\quad \partial S^mc=S^m\partial c=0.$$On en déduit alors que $S_mc$ est un cycle, d'où $S^mc\in H_n^\mathcal{U}(X)$. On en déduit dès lors que $i_\ast$ est surjectif.

Soit $[c]\in H_n(X)$, avec $c\in C_n^\mathcal{U}(X)$ tel que $\partial c=0$, et l'on veut montrer que~$i_\ast([c])=0$, c'est à dire qu'il existe $b\in C_{n+1}(X)$ tel que $c=\partial b$. Nous savons qu'il existe $m$ tel que $S^mb\in C_{n+1}^\mathcal{U}(X)$. En notant $b'=S^mb+T_mc\in C_{n+1}^\mathcal{U}(X)$, nous avons : \[
\partial b'=\partial S^mb+\partial T_mc=S^mc+c-S^mc-T_m\partial c=c\]dont nous pouvons en conclure que $i_\ast$ est injectif.

\bigskip Enfin, voyons comment ce résultat permet de démontrer le théorème. Soient~$X$ un espace et~${Z\subset A\subset X}$ tel que $\overline{Z}\subset Int(A)$. On choisi $\mathcal{U}=\{Int(A),X\setminus Z\}$, qui est un recouvrement puisque $Int(A)\cup Int(X\setminus Z)=Int(A)\cup X\setminus Int(Z)$. La composition suivante nous donne les mêmes résultats que pour les $\Delta$-complexes :$$C_n(X\setminus Z)\to C_n^\mathcal{U}(X)\to C_n^\mathcal{U}(X)/C_n^\mathcal{U}(A).$$
\end{proof}

Le premier résultat qui découle de ceci est sur le quotient.

\begin{proposition}
Soit $X$ un espace et soit $A$ un sous-espace fermé, qui est rétracte par déformation d'un voisinage dans $X$. Alors l'application quotient~$q:(X,A)\to (X/A,A/A)$ induit un isomorphisme $q_\ast$ entre $H_n(X,A)$ et~$H_n(X/A,A/A)$ pour tout $n$.
\end{proposition}

Un couple $(X,A)$ vérifiant la condition du théorème est appelée une \emph{bonne paire}.

\begin{proof}
Soit $U$ un voisinage qui se rétracte par déformation en $A$. Nous avons ce diagramme commutatif suivant : \[\small\begin{tikzcd}
H_n(X,A)\arrow[d,"q_\ast"]\arrow[r]&H_n(X,V)\arrow[d,"q_\ast"]&H_n(X\setminus A,V\setminus A)\arrow[d,"q_\ast"]\arrow[l]\\
H_n(X/A,A/A)\arrow[r]&H_n(X/A,V/A)&H_n(X/A\setminus A/A,V/A\setminus A/A)\arrow[l]
\end{tikzcd}\]Les deux flèches de gauches sont des isomorphismes du fait qu'elles sont extraites de séquences exactes pour lesquelles il y a un terme nul, du à la rétraction par déformation. Les deux autres s'obtiennent par le théorème d'excision. Le quotient $q_\ast$ de droite est un isomorphisme du fait que $q$ restreint au complémentaire de $A$ est un homéomorphisme.

Nous pouvons ainsi parcourir le diagramme avec des isomorphismes, allant d'en haut à gauche du bas à gauche.
\end{proof}

\begin{exemple}
Nous avions fini l'exemple des sphères avec~${\Tilde{H}_k(D^n,\partial D^n)\cong \Tilde{H}_{k-1}(\s{n-1})}$. En admettant le fait que $(D^n,\partial D^n)$ est une bonne paire, on obtient la relation suivante, avec la proposition sur le quotient : \[\Tilde{H}_k(\s{n})\cong \Tilde{H}_{k-1}(\s{n-1}).\]En sachant que $\Tilde{H}_0(\s{0})\cong\bb{Z}$ et $\Tilde{H}_k(\s{0})=0$ pour $k>0$ (du fait que $\s{0}$ est l'ensemble de 2 points distincts), nous obtenons par récurrence le résultat suivant : \[ \Tilde{H}_k(\s{n})\cong\left\{\begin{matrix}
\bb{Z}&si\ k=n\\
0&sinon.
\end{matrix}\right.\]
\end{exemple}

\subsubsection{Applications de l'excision}

Le premier résultat est ce qui a été appelé l'invariance de la dimension, qui stipule que $\bb{R}^n$ ne peut pas être homéomorphes à $\bb{R}^m$ si $n\neq m$.

\begin{theorem}
S'il existe deux ouverts $U\subset\bb{R}^n$ et $V\subset\bb{R}^m$ qui sont homéomorphes, alors $n=m$.
\end{theorem}
\begin{proof}
Supposons qu'il existe de tels ouverts $U$ et $V$. Tout d'abord, pour $x\in U$, nous avons~${H_k(U,U\setminus\{x\})\cong H_k(\bb{R}^n,\bb{R}^n\setminus\{x\})}$ par excision. Or $\bb{R}^n\setminus\{x\}$ se rétracte par déformation en~$\s{n-1}$. Nous obtenons la séquence exacte suivante : \[\begin{tikzcd}
0=H_k(\bb{R}^n)\arrow[r]&H_k(\bb{R}^n,\bb{R}^n\setminus\{x\})\arrow[r] &H_{k-1}(\s{n-1})\arrow[r]&H_k(\bb{R}^n)=0,
\end{tikzcd}\]d'où nous pouvons en déduire que $H_k(\bb{R}^n,\bb{R}^n\setminus\{x\})\cong H_k(\s{n-1})$. Son groupe d'homologie est $\bb{Z}$ pour $k=n$ et 0 sinon.

Nous pouvons calculer l'homologie de $H_k(V,V\setminus\{x'\})$, avec $x'\in V$ de la même manière. Étant donné que les groupes d'homologies sont invariant par homéomorphisme, on en déduit alors que le groupe d'homologie non trivial est le même, autrement dit que $n=m$.
\end{proof}

Pour un espace $X$ et un point $x\in X$, on nomme les groupes $H_n(X,X\setminus\{x\})$ comme étant les groupes d'\emph{homologie locale} de $X$ en $x$.

\bigskip Pour une variété $V$, l'homologie locale de $V$ en $x\in V$ permet de calculer la dimension de l'espace euclidien auquel le voisinage de $x$ est homéomorphe. En effet, en notant $U$ un voisinage de $x$ qui est homéomorphe à $\bb{R}^n$, nous obtenons les isomorphismes suivant, frâce au théorème d'excision : \[H_k(V,V\setminus\{x\})\cong H_k(U,U\setminus\{x\})\cong H_k(\bb{R}^n,\s{n-1})\cong\left\{\begin{matrix}
\bb{Z}&\text{si } k=n\\
0&\text{sinon}.
\end{matrix}\right.\]

\subsection{Simpliciale et singulière : même combat}

Le résultat que nous allons voir ici est une grande consécration dans le monde de l'homologie. Nous avons vu que les homologies simpliciales et singulières étaient toutes les deux différentes, avec chacune leurs qualités et leurs défauts. Là où la simpliciale est utile concrètement car elle permet de calculer efficacement, l'autre permet d'obtenir des résultats théorique importants. Si l'on arrivait à unifier ces deux homologies, nous aurions en notre possession un outil puissant, permettant d'obtenir tous les avantages de ces deux homologies. En plus de cela, cela permettrait de dire que les groupes d'homologie d'un $\Delta$-complexe sont indépendant de la structure simpliciale considérée.

\bigskip Nous définissons l'homologie simpliciale relative de la même façon que nous avons fait pour l'homologie relative singulière. Pour $(X,A)$ une bonne paire de $\Delta$-complexes, il existe un morphisme canonique $\Delta_n(X,A)\to C_n(X,A)$, envoyant chaque simplexe sur leur application caractéristique~$\sigma$. Cela permet de définir un morphisme $\homsimp_n(X,A)\to H_n(X,A)$. On note $X^k$ l'ensemble des simplexes de dimensions $k$ ou inférieure constituant $X$. Ce qui implique alors deux choses : $X^{k-1}\subset X^k$, et~$X^k/X^{k-1}$ est l'ensemble des $k$ simplexes de $X$.

\begin{theorem}
Le morphisme $H_n^\Delta(X)\to H_n(X)$ sont des isomorphismes pour tout $n$, et pour tout $X$ étant un $\Delta$-complexe.
\end{theorem}

L'idée de la preuve est de considérer le diagramme suivant : \[\footnotesize\begin{tikzcd}
H_{n+1}^\Delta(X^k,X^{k-1})\arrow[d]\arrow[r]&H_{n}^\Delta(X^{k-1})\arrow[d]\arrow[r]&H_{n}^\Delta(X^{k})\arrow[d]\arrow[r]&H_{n}^\Delta(X^k,X^{k-1})\arrow[d]\arrow[r]&H_{n-1}^\Delta(X^{k-1})\arrow[d]\\
H_{n+1}(X^k,X^{k-1})\arrow[r]&H_{n}(X^{k-1})\arrow[r]&H_{n}(X^{k})\arrow[r]&H_{n}(X^k,X^{k-1})\arrow[r]&H_{n-1}(X^{k-1}),
\end{tikzcd}\]et de montrer que le morphisme du centre est un isomorphisme. Pour cela, nous allons utiliser le lemme qui suit.

\subsubsection{Le lemme des cinq}
\begin{lemma}
Pour le diagramme commutatif qui suit, avec les lignes formant des séquences exactes, et $\alpha,\beta,\delta,\varepsilon$ des isomorphismes, alors $\gamma$ est un isomorphisme.
\[\begin{tikzcd}
A\arrow[d, "\alpha"]\arrow[r, "i"]&B\arrow[d, "\beta"]\arrow[r, "j"]&C\arrow[d, "\gamma"]\arrow[r, "k"]&D\arrow[d,"\delta"]\arrow[r, "l"]&E\arrow[d,"\varepsilon"]\\
A'\arrow[r, "i'"]&B'\arrow[r, "j'"]&C'\arrow[r, "k'"]&D'\arrow[r, "l'"]&E'.
\end{tikzcd}\]
\end{lemma}
\begin{proof}
Cette preuve se fait par diagramme chasing, en deux étapes : l'injectivité, puis la surjectivité.

Montrons que $\ker\gamma=0$. Soit $c\in \ker\gamma$. Si $c\in \ker k=\im j$, alors il existe $b\in B$ tel que $j(b)=c$. Par commutativité du diagramme, nous avons $\gamma j(b)=j'\beta(b)=0$, c'est à dire $\beta(b)\in\ker j'=\im i'$. Il existe alors $a'\in A'$ tel que $i'(a')=\beta(b)$. Or $\alpha$ est un isomorphisme, on en déduit alors qu'il existe~$a\in A$ tel que $i(a)=b$. On en déduit que $c=0$, puisque $b\in \im i=\ker j$. Si $c\notin\ker k$, alors il  existe $d\in D$ tel que $d=k(d)$. Par injectivité de $\delta$, il existe $d'\in D'$ tel que $d'=\varepsilon(d)=\varepsilon k(c)$. Or~$\varepsilon k(c)=k'\gamma(c)=k'(0)$. Nous en déduisons alors que $\gamma(c)\in\ker k'=\im j'$. Il existe donc $b'\in B'$ tel que $\gamma(c)=j'(b)$. Du fait que $\beta$ soit un isomorphisme il existe $b\in B$ tel que $j(b)=c$. Nous avons alors $c\in\im j=\ker k$, ce qui est absurde. Nous en concluons alors que $c\in \ker k'$, et donc que $c=0$.

Soit $c'\in C'$, montrons qu'il existe $c\in C$ tel que $\gamma(c)=c'$. Si $c'\in\ker k'=\im j'$, il existe $b'\in B'$ tel que $j'(b')=c'$. Or $\beta$ est un isomorphisme, il existe donc $b\in B$ tel que $\gamma(j(b))=j'\beta j=c'$, ce que nous voulions. Si $c'\notin\ker k'$, alors il existe $d'\in D'$ tel que $k'(c')=d'$. Comme $\delta$ est un isomorphisme, il existe $d\in D$ tel que $\delta(d)=d'=k'(c')$. De plus, nous avons $l'(d')=0$ car $d'\in\ker l'=\im k'$. Or~$\varepsilon$ est un isomorphisme, alors $\varepsilon l(d)=0$, et donc $l(d)=0$. Nous en déduisons que $d\in\ker l=\im k$, et donc qu'il existe $c$ tel que $k(c)=d$. Ce $c$ vérifie, par commutativité du diagramme, $\gamma(c)=c'$.
\end{proof}

\subsubsection{Démonstration du théorème}

\begin{proof}
Montrons dans un premier temps que les groupes $\homsimp_n(X^k,X^{k-1})$ et $H_n(X^k,X^{k-1})$ sont isomorphes. Nous avons $\Delta_n(X^k,X^{k-1})=\Delta_n(X^k)/\Delta_n(X^{k-1})$. Si $n>k$, alors $\Delta_n(X^k)=0$. Si $n<k$, alors $\Delta_n(X^k)=\Delta_n(X^{k-1})$, correspondant au groupe libre abélien avec comme base l'ensemble des $n$-simplexes de $X$. Enfin, si $n=k$, nous avons $\Delta_n(X^k)$ le groupe libre abélien formé par l'ensemble des $k$-simplexes, et $\Delta_n(X^{k-1})=0$. Nous pouvons alors en conclure le résultat suivant : \[\Delta_n(X^k,X^{k-1})= \left\{\begin{matrix}
\bb{Z}^l & \text{si $n=k$, l=$\#\{k$-simplexes de X$\}$}\\
0 & \text{sinon}. \\
\end{matrix}\right.\]Ainsi, les morphismes $\partial$ sont tous triviaux, de telle sorte que les groupes d'homologie simpliciale soient définis de manière identique.

D'un autre côté, nous avons le couple $(X^k,X^{k-1})$ qui forme une bonne paire. Nous avons alors~${H_n(X^k,X^{k-1})\cong H_n(X^k/X^{k-1})}$. Or $X^k/X^{k-1}$ est l'ensemble des $k$-simplexes de $X$. On en déduit alors le résultat suivant : \[C_n(X^k/X^{k-1})=\left\{\begin{matrix}
\bb{Z}^l & \text{si $n=k$, l=$\#\{k$-simplexes de X$\}$}\\
0&\text{sinon.}
\end{matrix}\right.\]Pour la même raison que pour l'homologie simpliciale, les groupes d'homologies sont définis de la même manière.

Nous pouvons alors en conclure $\homsimp_n(X^k,X^{k-1})\cong H_n(X^k,X^{k-1})$, pour tout~$n$, et donc que sur le diagramme la première et quatrième colonne sont des isomophismes.

\bigskip Ensuite, nous avons $\Delta_n(X^0)$ qui vaut 0 si $n>0$, et le groupe libre abélien de base les 0-simplexes de $X$ pour $n=0$.Le passage à l'homologie identifie les points connectés entre eux, de telle sorte $\homsimp_0(X^0)=\bb{Z}^l$ avec $l$ le nombre de composante connexes. C'est exactement la même définition que pour $H_0(X^0)$, nous avons alors un isomorphisme entre $\homsimp_0(X^0)$ et $H_0(X^0)$. Les groupes d'homologies supérieurs, que ce soit simpliciales ou singulières, sont tous triviaux. Nous avons alors un isomorphisme entre les groupes $\homsimp_n(X^0)$ et $H_n(X^0)$, pour tout $n$.

Par récurrence sur $k$, nous supposons que les groupes $\homsimp_n(X^k)$ et~$H_n(X^k)$ sont isomorphes, pour tout $n$. Nous avons ainsi des isomorphismes sur les colonnes 2 et 5, ce qui nous permet maintenant d'appliquer le lemme des cinq pour en déduire que la colonne du centre est un isomorphisme.

Nous avons ainsi montrer que $\homsimp_n(X^k)\cong H_n(X^k)$, pour tout $k$ et tout $n$. En notant $K$ la dimension maximale des simplexes utilisés pour la structure de~$X$, nous avons $X^K=X$. Nous pouvons désormais conclure qu'il existe bien une équivalence entre l'homologie simpliciale et l'homologie singulière.
\end{proof}

Nous pouvons facilement démontrer l'équivalence des homologies relatives.

\begin{corollary}
Le morphisme $\homsimp_n(X,A)\to H_n(X,A)$ est un isomorphisme, pour tout $n$ et pour toute paire de $\Delta$-complexes $(X,A)$, avec $A\subset X$.
\end{corollary}

\begin{proof}
On considère le diagramme commutatif suivant, composé de deux séquences exactes : \[\begin{tikzcd}
\homsimp_n(A)\arrow[r]\arrow[d]&\homsimp_n(X)\arrow[r]\arrow[d]&\homsimp_n(X,A)\arrow[r]\arrow[d]&\homsimp_{n-1}(A)\arrow[r]\arrow[d]&\homsimp_{n-1}(X)\arrow[d]\\
H_n(A)\arrow[r]&H_n(X)\arrow[r]&H_n(X,A)\arrow[r]&H_{n-1}(A)\arrow[r]&H_{n-1}(X).
\end{tikzcd}\]Comme $X$ et $A$ sont des $\Delta$-complexes, le théorème nous dit que les deux premières et deux dernières colonnes sont des isomorphismes. Le lemme des cinq nous permet donc de conclure que le morphisme~${\homsimp_n(X,A)\to H_n(X,A)}$ est un isomorphisme.
\end{proof}

\subsection{Abélianisation du groupe fondamental}\label{sect:abelianisation}

Nous pouvons voir les chemins comme des 1-simplexes, du fait que $[0,1]$ et~$\Delta^1$ sont homéomorphes. Avec cette idée, nous pouvons voir les lacets comme des cycles, car ${\partial\gamma=\gamma(1)-\gamma(0)=0}$. Nous allons montrer ici que le groupe $H_1$ est en réalité l'abélianisation du groupe fondamental.

\subsubsection{Lien entre chemins et simplexes}

Pour que l'affirmation soit vraie, il faut tout d'abord vérifier que l'on puisse définir un morphisme entre le groupe fondamental vers le premier groupe d'homologie.

\begin{proposition}
En considérant les lacets comme des 1-simplexes singulier, il existe un morphisme $\pi_1(X,x_0)\to H_1(X)$, pour tout espace $X$ et $x_0\in X$.
\end{proposition}

\begin{proof}
Dans un premier temps, montrons que le chemin constant $c$ est homologue à 0, c'est à dire une bordure. Étant constant, il est un lacet, donc il peut être considéré comme un cycle. Le~2-simplexe constant possède ce cycle sur chaque bordure, ce qui donne la formule $\partial\sigma=c-c+c=c$. Autrement dit,~$c\in\im\partial$, donc $c\sim0$.

\begin{figure}[H]
\centering
\begin{subfigure}{.45\linewidth}
\centering
\begin{tikzpicture}[scale=.6]
    % carré
    \fill[gray!30] (0,0) rectangle (4,4);
    \begin{scope}[thick,decoration={markings, mark=at position 0.5 with {\arrow{stealth}}}] 
    \draw[postaction={decorate}, red] (0,4)--(4,4)node[midway, below] {$\gamma'$};
    \draw[postaction={decorate}, red] (0,0)--(4,0) node[midway, above] {$\gamma$};
    \draw[postaction={decorate}] (0,0)--(0,4);
    \draw[postaction={decorate}] (4,0)--(4,4);
    \draw[postaction={decorate}, brown] (0,0) -- (4,4);
\end{scope}
    \filldraw[] (0,0) circle(0pt) node[left] {$\gamma(0)$};
    \filldraw[] (0,4) circle (0pt) node[left] {$\gamma'(0)$};
    \filldraw[] (4,0) circle (0pt) node[right] {$\gamma(1)$};
    \filldraw[] (4,4) circle (0pt) node[right] {$\gamma'(1)$};
    \filldraw[] (3,1) circle (0pt) node[] {$\sigma_1$};
    \filldraw[] (1,3) circle (0pt) node[] {$\sigma_2$};
\end{tikzpicture}
\caption{\centering Homotopie entre $\gamma$ et $\gamma'$}\label{tkz:homotopy-as-homology}
\end{subfigure}
\begin{subfigure}{.45\linewidth}
\centering
\begin{tikzpicture}[scale=0.5]
\begin{scope}[thick,decoration={markings, mark=at position 0.5 with {\arrow{stealth}}}] 
    \draw[postaction={decorate}] (0,0)--(3,4)node[midway, above left]{$\gamma\cdot\gamma'$};
    \draw[postaction={decorate}] (0,0) -- (6,0) node[midway, below]{$\gamma$};
    \draw[postaction={decorate}](6,0) -- (3,4) node[midway, above right]{$\gamma'$};
\end{scope}
    \draw[thick, dashed] (6,0) -- (1.5,2);
    %points
    \filldraw[](0,0) circle(0pt) node[left]{$v_0$};
    \filldraw[](6,0) circle(0pt) node[right]{$v_1$};
    \filldraw[](3,4) circle(0pt) node[above]{$v_2$};
    % triangle
    \node at (3,2) {$\sigma$};
\end{tikzpicture}
\caption{Simplexe obtenu par composition}
\label{tkz:simplex-for-cdot}
\end{subfigure}
\end{figure}

Montrons que la relation d'homotopie implique la relation d'homologie, prouvant alors que le morphisme ne dépende pas du représentant. Soient~$\gamma,\gamma'$ deux chemins de $X$, tel que $\gamma\simeq \gamma'$. Il existe une homotopie $\gamma_t$ entre les deux lacets, telle que $\gamma_0=\gamma$ et $\gamma_1=\gamma'$. Nous pouvons alors construire la figure \ref{tkz:homotopy-as-homology}, en considérant les 2-simplexes $\sigma_1=[\gamma(0),\gamma(1),\gamma'(1)]$ et $\sigma_2=[\sigma(0), \sigma'(0),\sigma'(1)]$. En remarquant que $\gamma(0)=\gamma'(0)$ et $\gamma(1)=\gamma'(1)$. nous pouvons effectuer le calcul suivant : \[\begin{split}
\partial(\sigma_1-\sigma_2)&=[\gamma(1),\gamma'(1)]-[\gamma(0),\gamma'(1)]+\gamma-\gamma'+[\gamma(0),\gamma'(1)]-[\gamma(0),\gamma'(0)]\\
&=0-[\gamma(0),\gamma'(1)]+\gamma-\gamma'+[\gamma(0),\gamma'(1)]-0\\
&=\gamma-\gamma'.
\end{split}\]On en déduit alors $\gamma-\gamma'\sim 0$, c'est à dire $\gamma\sim \gamma'$.

\bigskip Montrons que la loi de composition de chemin est compatible avec la loi additive de simplexes. Soient $\gamma,\gamma'$ deux chemins tels que $\gamma(1)=\gamma'(0)$, de telle sorte que l'on puisse définir $\gamma\cdot\gamma'$. On considère $\sigma$ la composition de la projection orthogonale de $\Delta^2=[v_0,v_1,v_2]$ sur le côté $[v_0,v_2]$, suivi de $\gamma\cdot\gamma':[v_0,v_2]\to X$. Nous obtenons la figure \ref{tkz:simplex-for-cdot}, de telle sorte que $\partial \sigma=\gamma-\gamma\cdot\gamma'+\gamma'$. On peut alors en conclure que $\gamma\cdot\gamma'\sim \gamma+\gamma'$.

\bigskip Enfin, montrons que le chemin inverse est la négation d'un simplexe. Soit~$\gamma$ un chemin, montrons que $\overline{\gamma}\sim -\gamma$. Comme $\gamma\cdot\overline{\gamma}$ est homotope au chemin constant, nous avons $\gamma\cdot\overline{\gamma}\sim 0$. Avec le point précédent, nous avons $\gamma\cdot\overline{\gamma}\sim \gamma+\overline{\gamma}$. Nous pouvons alors en déduire que $\overline{\gamma}\sim -\gamma$.

\bigskip Sachant que tous les points restent vrais pour les lacets, nous pouvons définir un morphisme du groupe fondamental vers le premier groupe d'homologie.
\end{proof}

Maintenant que nous avons montré que l'on peut créer un morphisme de groupes, nous allons voir que l'on peut induire un isomorphisme entre l'abélianisé du groupe fondamental et le premier groupe d'homologie.

\subsubsection{Isomorphisme}


\begin{theorem}
Le morphisme $h:\pi_1(X,x_0)\to H_1(X)$, qui à une classe d'homotopie $[\gamma]_{\pi_1}$ renvoie la classe d'homologie $[\gamma]_H$, est surjectif et induit un isomorphisme entre l'abélianisé de $\pi_1(X,x_0)$ et~$H_1(X)$, pour tout espace $X$ est connexe par arc, et $x_0\in X$.
\end{theorem}

\begin{proof}
Pour montrer l'isomorphisme, nous allons montrer que le noyau de $h$ est le groupe dérivé du groupe fondamental $[\pi_1,\pi_1]$, permettant de conclure avec le premier théorème d'isomorphisme. Mais avant cela, montrons que $h$ est surjectif.

Soit $[\sigma]\in H_1(X)$, avec $\sigma$ un 1-cycle, tel que $\sigma=\sum_in_i\sigma_i$, pour $\sigma_i:[0,1]\to X$ des 1-simplexes. Étant donné que $\sigma$ est un cycle, nous avons : \[0=\partial\sigma=\sum_in_i\partial\sigma_i=\sum_in_i\big(\sigma_i(1)-\sigma_i(0)\big)=\sum_{p\in S}m_pp,\]avec $S=\{\sigma_i(0),\sigma_i(1),\forall i\}$. La dernière égalité implique que $m_p=0$ pour tout~$p\in S$. Nous pouvons supposer, par continuité de $\sigma$, que les $\sigma_i$ sont rangés dans l'ordre, tel que $\sigma_i(1)=\sigma_{i+1}(0)$. Comme~$X$ est connexe par arc, nous pouvons construire un chemin entre $x_0$ et $p\in X$, que nous noterons $\beta_p$. Ainsi, nous pouvons construire un lacet parcourant $\sigma_i$, par $\eta_i=\beta_{\sigma_i(0)}\cdot\sigma_i\cdot\overline{\beta_{\sigma_i(1)}}$. Nous pouvons décider d'avoir $\beta_{\sigma_i(1)}=\beta_{\sigma_{i+1}(0)}$, de telle sorte que la concaténation donne~$\eta_i\cdot\eta_{i+1}\simeq \beta_{\sigma_i(0)}\cdot\sigma_i\cdot\sigma_{i+1}\cdot\overline{\beta_{\sigma_{i+1}(1)}}$. Nous pouvons alors construire le lacet~$\prod_i\eta_i^{n_i}$, de telle sorte qu'il vérifie : \[\begin{split}
h\Bigg(\Big[\prod_i\eta_i^{n_i}\Big]_{\pi_1}\Bigg)&=\Big[\sum_in_i\eta_i\Big]_H\\
&=\Big[\sum_in_i\big(\beta_{\sigma_i(0)}+\sigma_i-\beta_{\sigma_i(1)}\big)\Big]_H\\
&=\Big[\sum_in_i\sigma_i+\sum_in_i(\beta_{\sigma_i(0)}-\beta_{\sigma_i(1)})\Big]_H\\
&=\Big[\sigma+\sum_{p\in S}m_p\beta_p\Big]_H=[\sigma]_H.
\end{split}\]Nous venons alors de construire un lacet tel que l'image de sa classe d'homotopie est la classe d'homologie de $\sigma$. Cela démontre alors que $h$ est surjectif.

\bigskip Montrons désormais l'égalité entre le noyau et le groupe dérivé, que nous noterons $[\pi_1,\pi_1]$. Pour cela, nous allons utiliser le fait que pour un mot du groupe dérivé, chaque terme est exprimé avec des puissances, telles que la somme de celles-ci vaut 0, puisque par la projection dans l'abélianisation, le mot vaut 0. Du fait que $H_1(X)$ est abélien, tout élément du groupe dérivé est donc envoyé sur 0. Nous avons alors $[\pi_1,\pi_1]\subset \ker h$. Il nous reste alors à montrer l'inclusion inverse~$\ker h\subset [\pi_1,\pi_1]$.

On considère $[\gamma]_{\pi_1}\in\ker h$. Avec ce que l'on a vu dans cette section, cela revient à dire que~$\gamma\sim 0$, donc que c'est une bordure. Il existe alors un~2-simplexe~$\sigma=\sum_in_i\sigma_i$ tel que $\partial\sigma=\gamma$. En notant la bordure de chaque~$\sigma_i$ comme étant la somme des 1-simplexes $\lambda_i-\mu_i+\nu_i$, nous avons également : \[\gamma=\partial\left(\sum_in_i\sigma_i\right)=\sum_in_i\partial\sigma_i=\sum_in_i(\lambda_i-\mu_i+\nu_i).\]Nous noterons $S:=\{\lambda_i,\mu_i,\nu_i,\forall i\}$, en faisant en sorte que le simplexe $\gamma$ apparaisse dedans. En notant $m_\theta$ la somme des coefficients de $\theta\in S$ apparaissant dans la dernière somme $\sum_in_i(\lambda_i-\mu_i+\mu_i)$, nous avons $m_\theta=0$ si $\theta\neq\gamma$, ainsi que $m_\gamma=1$.

De la même manière que pour montrer la surjectivité, nous pouvons définir un lacet traversant chaque simplexe. En réutilisant la notation $\beta_p$ pour le chemin partant de $x_0$ allant vers $p$, nous pouvons construire les lacets $\beta_{\lambda_i(0)}\cdot\lambda_i\cdot\overline{\beta_{\lambda_i(1)}}$, idem pour $\mu_i$ et$\nu_i$. Nous noterons alors $\eta_i$ la concaténation de ces lacets, représenté figure \ref{fig:eta_i}, qui est donné par : \[\eta_i=\big(\beta_{\lambda_i(0)}\cdot\lambda_i\cdot\overline{\beta_{\lambda_i(1)}}\big)\cdot\big(\beta_{\mu_i(1)}\cdot\overline{\mu_i}\cdot\overline{\beta_{\mu_i(0)}}\big)\cdot\big(\beta_{\nu_i(0)}\cdot\nu_i\cdot\overline{\beta_{\nu_i(1)}}\big).\] Ces lacets sont homotopes au lacet constant, puisque $\sigma_i$ est contractible. Nous avons alors~${\prod_i\eta_i^{n_i}\sim0}$, et donc par extension $\big(\prod_i\eta_i^{n_i}\cdot\gamma\inv\big)\sim\gamma\inv$. D'un autre côté, pour tout $\theta\in S$, nous avons $\beta_{\theta(0)}\cdot\theta\cdot\overline{\beta_{\theta(1)}}$ qui apparaît exactement~$m_\theta$ fois dans le terme $\prod_i\eta_i^{n_i}$. Nous pouvons alors déduire que ${\prod_i\eta_i^{n_i}\simeq\gamma}$, ce qui veut dire que l'on a : \[\gamma\inv\sim\left(\prod_i\eta_i^{n_i}\right)\cdot\gamma\inv\sim\gamma\cdot\gamma\inv\sim 0.\]On en conclut donc que $[\gamma\inv]_{\pi_1}\in[\pi_1,\pi_1]$, et donc par stabilité de l'inverse, que~$[\gamma]_{\pi_1}\in[\pi_1,\pi_1]$.
\end{proof}

\begin{figure}[H]
\centering
\begin{tikzpicture}[scale=1.8]
\begin{scope}[thick,decoration={markings, mark=at position 0.5 with {\arrow{stealth}}}] 
%simplex
\draw[postaction={decorate}] (2,2)--(4,1.5) node[midway, above]{$\lambda_i$};
\draw[postaction={decorate}] (3,3.5)--(2,2) node[midway, left]{$\nu_i$};
\draw[postaction={decorate}] (3,3.5)--(4,1.5) node[midway, right]{$\mu_i$};
%paths
%above
\draw[postaction={decorate}, dashed, red!80] (0,2) .. controls (1.5,3.2) .. (3,3.5) node[midway, above left]{$\overline{\beta_{\mu_i(0)}}$};
\draw[postaction={decorate}, dashed, red!80] (3,3.5) .. controls (1.5,3.3) .. (0,2) node[midway, below right]{$\beta_{\nu_i(0)}$};

%mid
\draw[postaction={decorate}, dashed, red!80] (0,2) .. controls (1, 2.1) .. (2,2) node[midway, above]{$\beta_{\lambda_i(0)}$};
\draw[postaction={decorate}, dashed, red!80] (2,2) .. controls (1,2) .. (0,2) node[midway, below]{$\overline{\beta_{\nu_i(1)}}$};

%below
\draw[postaction={decorate}, dashed, red!80] (0,2) .. controls (1.5,1.4) .. (4,1.5) node[midway, above right]{$\beta_{\mu_i(1)}$};
\draw[postaction={decorate}, dashed, red!80] (4,1.5) .. controls (1.5,1.3) .. (0,2) node[midway, below right]{$\overline{\beta_{\lambda_i(1)}}$};
\end{scope}
\filldraw[] (0,2) circle(.7pt) node[below]{$x_0$};
\filldraw[thick] (3,2.5) circle(0pt) node[]{$\sigma_i$};
\end{tikzpicture}
\caption{Construction du lacet $\eta_i$}
\label{fig:eta_i}
\end{figure}